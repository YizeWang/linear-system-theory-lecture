% style settings
\documentclass[a4paper,10.5pt]{article}
\headsep 16mm
\parindent 0mm
\headheight 8mm
\topmargin -20mm
\textwidth 160mm
\textheight 240mm
\baselineskip 7mm
\oddsidemargin -.15in

% import packages
\usepackage{color}
\usepackage{epsfig}
\usepackage{psfrag}
\usepackage{amsmath}
\usepackage{amssymb}
\usepackage{theorem}
\usepackage{multicol}
\usepackage{mathtools}

% define shortcuts
\def\dim{{\mbox{\em dim}}}
\def\ifpart{\textbf{(if) }}
\def\rank{{\mbox{\sc Rank}}}
\def\nullspace{{\mbox{\sc Null}}}
\def\range{{\mbox{\sc Range}}}
\def\domain{{\mbox{\em domain}}}
\def\nullity{{\mbox{\sc Nullity}}}
\def\onlyifpart{\textbf{(only if) }}
\def\solution{\noindent{\it Solution.}}

% define new commands and environment
\newtheorem{lemma}{Lemma}
\newtheorem{proof}{Proof}
\newcommand{\de}{\mathrm{d}}
\newcommand{\R}{\mathbb{R}}
\newcommand{\N}{\mathbb{N}}
\newcommand{\I}{\mathbb{I}}
\newcommand{\A}{\mathcal{A}}
\newcommand{\B}{\mathcal{B}}
\newcommand{\defeq}{\vcentcolon=}
\newcommand{\dt}{\, \mathrm{d} t}
\newcommand{\dx}{\, \mathrm{d} x}
\newcommand{\supx}{\sup_{x\neq0}}
\newcommand{\abs}[1]{\left|#1\right|}
\newcommand{\spanop}{\mathop{\rm span}}
\newcommand{\norm}[1]{\left\lVert#1\right\rVert}
\newenvironment{mymat}{\left[\begin{matrix}}{\end{matrix}\right]}

\begin{document}
\thispagestyle{plain}

% title
\vspace*{-1.5cm} {\noindent \rule{15.8cm}{.3mm} \\[.3cm]}
\begin{center} \bf
{\large Linear System Theory \medskip \\
Problem Set 3 \\
Linear Time-Varying Systems, Linear Time-Invariant Systems \medskip
\\ Issue date: Oct. 21, 2019 \\ Due date: Oct. 31, 2019}
\end{center}
\rule{15.8cm}{.3mm} \\[1cm]

% exercise 1
\noindent {\bf Exercise 1. (Linear Time-Varying Systems, $[$50 points in total$]$)}

Let $A_1(t), A_2(t)$, $F(t) \in \R^{n \times n}$ be piecewise continuous matrix functions. Let $\Phi_i$ be the state transition matrix for $\dot{x}(t) = A_i(t)x(t)$, for $i = 1,2$. Consider the matrix differential equation:
\begin{align*}
\dot{X}(t) = A_1(t)X(t) + X(t) A^T_2(t) + F(t), \; X(t_0) = X_0,
\end{align*}
where $X(t)\in \R^{n\times n}$ for any $t\ge t_0$.

\begin{enumerate}
\item $[$20 {\bf points}$]$ Show that this is an affine time-varying system.  (Hint: An affine time-varying system is a system of the form $\dot{x}(t) = A(t)x(t) + b(t)$, where $x(t)$ and $b(t)$ are vectors.)
\item $[$30 {\bf points}$]$ Assume that the solution of the above system can be written as:
\begin{align*}
X(t) = \Phi_1(t,t_0) X_0 \Phi^T_2(t,t_0) + \int_{t_0}^{t}\Phi_1(t,\tau)M(t,\tau)d\tau.
\end{align*}
Express the matrix $M(t,\tau)$ as a function of $\Phi_1(t,\tau)$, $F(t)$, and $\Phi_2(t,\tau)$. (Hint: $\Phi_1(t,\tau)$, $F(t)$, and $\Phi_2(t,\tau)$ may not all appear in the expression of $M(t,\tau)$.)
\end{enumerate}\par

\noindent {\bf Solution 1.1} \\

Let $x_{ij},a^{(1)}_{ij},a^{(2)}_{ij},f_{ij}$ denote the $i^{th}$ row and $j^{th}$ column element of matrix $X,A_1,A_2,F$ respectively. According to the matrix differential equation, $x_{ij}$ can be expressed as
\begin{align}
\label{eq:elementwise_expression}
\dot{x}_{ij} &= \sum_{k=1}^{n} a^{(1)}_{ik}x_{kj} + \sum_{k=1}^{n}x_{ik}a^{(2)}_{jk} + f_{ij} \nonumber \\
&= \sum_{k=1}^{n} \left(a^{(1)}_{ik}x_{kj} + a^{(2)}_{jk}x_{ik}\right) + f_{ij}
\end{align}
Let $x_{ci}$ and $f_{ci}$ denote the $i^{th}$ column of matrix $X$ and $F$ respectively. Define vector $\tilde{x},\tilde{f}\in \R^{n^2}$ by rearranging $x_{ci}$ and $f_{ci}$:
\begin{align*}
\tilde{x} = \begin{bmatrix}x_{c1} \\ \vdots \\ x_{cn}\end{bmatrix}, \quad
\tilde{f} = \begin{bmatrix}f_{c1} \\ \vdots \\ f_{cn}\end{bmatrix} \; .
\end{align*}
Based on the equation \eqref{eq:elementwise_expression}, elements of $\dot{\tilde{x}}$ can be expressed as
\begin{align}
\label{eq:linear_combination}
\dot{\tilde{x}}_{i+n(j-1)} = \sum_{k=1}^{n} a^{(1)}_{ik}\tilde{x}_{k+n(j-1)} + \sum_{k=1}^{n}a^{(2)}_{jk}\tilde{x}_{i+n(k-1)} + f_{ij} \; ,
\end{align}
which defines an affine time-varying system. To express it more explicitly, we define augmented matrix $\tilde{A_1}$ and $\tilde{A_2}$ to facilitate further notation:
\begin{align*}
\tilde{A}_1 &= \text{diag}(A, \dots, A) = \begin{bmatrix}
A &        &   \\
  & \ddots &   \\
  &        & A
\end{bmatrix} \in \R^{n^2 \times n^2} \; , \\
\tilde{A}_2 &= \begin{bmatrix}
\text{diag}(a^{(2)}_{11}) & \text{diag}(a^{(2)}_{12}) & \cdots &\text{diag}(a^{(2)}_{1n}) \\
\text{diag}(a^{(2)}_{21}) & \text{diag}(a^{(2)}_{22}) & \cdots &\text{diag}(a^{(2)}_{2n}) \\
\vdots & \vdots & \ddots & \vdots \\
\text{diag}(a^{(2)}_{n1}) & \text{diag}(a^{(2)}_{n2}) & \cdots &\text{diag}(a^{(2)}_{nn}) \\
\end{bmatrix} \in \R^{n^2 \times n^2} \; ,
\end{align*}
where $\text{diag}(a^{(2)}_{ij})$ denotes an $n \times n$ diagonal matrix whose diagonal elements are all $a^{(2)}_{ij}$. Then the system can be fully described by an affine time-varying system in the standard form:
\begin{align*}
\dot{\tilde{x}}(t) = \left(\tilde{A}_1(t)+\tilde{A}_2(t)\right)\tilde{x}(t)+\tilde{f}(t)
\end{align*} \par

\noindent {\bf Solution 1.2} \\

\begin{flalign*}
\frac{\de}{\de t}\left[\int_{t_0}^{t}\Phi_1(t,\tau)M(t,\tau)\de\tau\right] 
&= \int_{t_0}^{t}\frac{\de}{\de t}\left(\Phi_1(t,\tau)M(t,\tau)\right)\de\tau + \Phi_1(t,t)M(t,t) \cdot 1 &\\
&= \int_{t_0}^t A_1(t) \Phi_1(t,\tau) M(t,\tau)\de\tau + \int_{t_0}^t \Phi_1(t,\tau)\frac{\de}{\de t}M(t,\tau)\de \tau + M(t,t) &\\
&= A_1(t)\int_{t_0}^t \Phi_1(t,\tau) M(t,\tau)\de\tau + \int_{t_0}^t \Phi_1(t,\tau)\frac{\de}{\de t}M(t,\tau)\de \tau + M(t,t) &
\end{flalign*}
\begin{flalign*}
\dot{X}(t) &= A_1(t)\Phi_1(t,t_0)X_0\Phi^T_2(t,t_0) + \Phi_1(t,t_0)X_0\Phi^T_2(t,t_0)A_2^T(t) + \frac{\de}{\de t}\left[\int_{t_0}^{t}\Phi_1(t,\tau)M(t,\tau)\de\tau\right] &\\
&=A_1(t)X(t) - A_1(t)\int_{t_0}^{t}\Phi_1(t,\tau)M(t,\tau)\de\tau + X(t)A_2^T(t) - \left( \int_{t_0}^{t}\Phi_1(t,\tau)M(t,\tau)\de\tau \right) A_2^T(t) &\\
&\phantom{=}+ A_1(t)\int_{t_0}^t \Phi_1(t,\tau) M(t,\tau)\de\tau + \int_{t_0}^t \Phi_1(t,\tau)\frac{\de}{\de t}M(t,\tau)\de \tau + M(t,t) &\\
&=A_1(t)X(t)+X(t)A_2^T(t)-\left( \int_{t_0}^{t}\Phi_1(t,\tau)M(t,\tau)\de\tau \right) A_2^T(t)+\int_{t_0}^t \Phi_1(t,\tau)\frac{\de}{\de t}M(t,\tau)\de \tau + M(t,t) &
\end{flalign*}
From the matrix differential equation, we also know that
\begin{align*}
\dot{X}(t) = A_1(t)X(t) + X(t) A^T_2(t) + F(t)
\end{align*}
Compare these two equations, it follows immediately
\begin{align*}
F(t) = -\left( \int_{t_0}^{t}\Phi_1(t,\tau)M(t,\tau)\de\tau \right) A_2^T(t)+\int_{t_0}^t \Phi_1(t,\tau)\frac{\de}{\de t}M(t,\tau)\de \tau + M(t,t)
\end{align*}
Guess a solution for $M(t,\tau)$:
\begin{align*}
M(t,\tau) = F(\tau)\Phi_2^T(t,\tau)
\end{align*}
The right-hand side would thus be
\begin{align*}
R &= -\left( \int_{t_0}^{t}\Phi_1(t,\tau)M(t,\tau)\de\tau \right) A_2^T(t)+\int_{t_0}^t \Phi_1(t,\tau)\frac{\de}{\de t}M(t,\tau)\de \tau + M(t,t) \\
&= -\left( \int_{t_0}^{t}\Phi_1(t,\tau)M(t,\tau)\de\tau \right) A_2^T(t)+\int_{t_0}^t \Phi_1(t,\tau)F(\tau)\Phi_2^T(t,\tau)A_2^T(t)\de \tau + F(t)\Phi_2^T(t,t) \\
&= F(t) \\
&= L
\end{align*}
Therefore, $X(t) = \Phi_1(t,t_0) X_0 \Phi^T_2(t,t_0) + \int_{t_0}^{t}\Phi_1(t,\tau)F(\tau)\Phi_2^T(t,\tau)\mathrm{d}\tau$ is a solution to the differential equation. Next we prove the uniqueness of the solution. \\

Define $g(X) = \dot{X}$, $\forall X_1,X_2 \in \R^{n \times n}, \forall t \geq t_0$,
\begin{align*}
\norm{g(X_1)-g(X_2)} &= \norm{\dot{X}_1-\dot{X}_1} \\
&=\norm{\left(A_1X_1 + X_1 A^T_2 + F\right)-\left(A_1X_2 + X_2 A^T_2 + F\right)} \\
&=\norm{A_1X_1-A_1X_2+ X_1 A^T_2- X_2A^T_2} \\
&=\norm{A_1(X_1-X_2)+ (X_1-X_2)A^T_2} \\
&\leq \norm{A_1(X_1-X_2)} + \norm{(X_1-X_2)A^T_2} \\
&\leq \norm{A_1}\norm{X_1-X_2} + \norm{X_1-X_2}\norm{A^T_2} \\
&= (\norm{A_1}+\norm{A_2^T})\norm{X_1-X_2} \\
&\defeq K(t)\norm{X_1(t)-X_2(t)}
\end{align*}
Therefore, $g(X)$ is globally Lipschitz with respect to $X$ and piece-wise continuous with respect to $t$, which indicates the uniqueness of the solution to the matrix differential equation. Thus, the value of $\int_{t_0}^{t}\Phi_1(t,\tau)M(t,\tau)\de\tau$ is unique and $\forall t$
\begin{align*}
\int_{t_0}^{t}\Phi_1(t,\tau)F(\tau)\Phi_2^T(t,\tau)\mathrm{d}\tau = \int_{t_0}^{t}\Phi_1(t,\tau)M(t,\tau)\de\tau \; .
\end{align*}
However, we cannot make sure the uniqueness of $M(t,\tau)$. For example, any $\tilde{M}(t,\tau)$ defined as
\begin{align*}
\tilde{M}(t,\tau) = \begin{cases}
F(\tau)\Phi_2^T(t,\tau) & t \notin \mathcal{D} \\
K & t \in \mathcal{D}
\end{cases}
\end{align*}
where $K$ is any finite real number, and $\mathcal{D}$ stands for the discontinuity set of $\tilde{M}(t,\tau)$. Then, $\tilde{M}(t,\tau)$ is also a valid solution to the matrix differential equation since
\begin{align*}
\forall t \geq t_0, \int_{t_0}^{t}\Phi_1(t,\tau)M(t,\tau)\de\tau =  \int_{t_0}^{t}\Phi_1(t,\tau)\tilde{M}(t,\tau)\de\tau
\end{align*}


\clearpage

% exercise 2
\noindent {\bf Exercise 2. (Linear Time-Invariant Systems, $[$50 points in total$]$)}

Consider the following affine system:
\begin{equation*}
  \dot{x}(t) = A\, x(t) + 
  \begin{bmatrix}
0\\ 1\\ 0\end{bmatrix}\, \quad
  y(t) = \begin{bmatrix}1 & 1 & 0\end{bmatrix}\, x(t).
\end{equation*}
where $A \in \R^{3 \times 3}$.
The matrix $A$ has eigenvalues $\lambda_1 = -2$ with multiplicity~2, and $\lambda_2 = -1$ with multiplicity~1.
The eigenvalue $\lambda_1$ has an eigenvector $v_1 = \begin{bmatrix}0\\ 2\\ 1\end{bmatrix}$, and a generalized eigenvector $v_1' = \begin{bmatrix}-1\\ 0\\ 1\end{bmatrix}$.
The eigenvalue $\lambda_2$ has the eigenvector $v_2 = \begin{bmatrix}0\\ 1\\ 0\end{bmatrix}$.
\begin{enumerate}
	\item $[$20 points$]$ Find the matrix $A$.
	\item $[$20 points$]$ Calculate $\exp(At)$.
	\item $[$10 points$]$ Given $x(0)=[0\,\,0\,\,1]^{\top}$, compute $y(t)$.
\end{enumerate} \ \\

\noindent {\bf Solution 2.1}
\begin{align*}
Av_2=\lambda_2v_2 &\Rightarrow a_{12} = 0, a_{22} = -1, a_{23} = 0 &\\
Av_1=\lambda_1v_1 &\Rightarrow a_{13} = 0, a_{23} = -2, a_{33} = -2 &\\
\left(A-\lambda_1\I\right)v_1'=v_1 &\Rightarrow a_{11} = -2, a_{21} = 4, a_{31} = -1 &\\
A&=\begin{bmatrix}
-2 &  0 &  0 \\
-4 & -1 & -2 \\
-1 &  0 & -2 \\
\end{bmatrix}
\end{align*}

\noindent {\bf Solution 2.2}
\begin{align*}
T^{-1} &= \begin{bmatrix}
v_1 & v_1' & v_2
\end{bmatrix} = \begin{bmatrix}
0 & -1 & 0 \\
2 & 0 & 1 \\
1 & 1 & 0
\end{bmatrix}, \quad
T = \begin{bmatrix}
1 & 0 & 1 \\
-1 & 0 & 0 \\
-2 & 1 & -2
\end{bmatrix} \\
J_1 &= \begin{bmatrix}
-2 & 1  \\
& -2 \end{bmatrix}, \quad J_2 = -1, \quad
J = \begin{bmatrix}
-2 & 1 &  \\
 & -2 &  \\
 &  & -1
\end{bmatrix} \\
e^{J_1t} &= \begin{bmatrix}
e^{-2t} & te^{-2t} \\
0 & e^{-2t}
\end{bmatrix}, \quad e^{J_2t} = e^{-t}, \quad 
e^{Jt} = \begin{bmatrix}
e^{-2t} & te^{-2t} &\\
 & e^{-2t} & \\
& & e^{-t}
\end{bmatrix} \\
e^{At} &= T^{-1}e^{Jt}T = \begin{bmatrix}
e^{-2t} & 0 & 0\\
2e^{-2t}-2e^{-t}-2te^{-2t} & e^{-t} & 2e^{-2t}-2e^{-t} \\
-te^{-2t} & 0 & e^{-2t} \end{bmatrix}
\end{align*}

\clearpage

\noindent {\bf Solution 2.3}
\begin{align*}
\Phi(t,0) &= e^{At} = \begin{bmatrix}
e^{-2t} & 0 & 0\\
2e^{-2t}-2e^{-t}-2te^{-2t} & e^{-t} & 2e^{-2t}-2e^{-t} \\
-te^{-2t} & 0 & e^{-2t} \end{bmatrix}, \quad B(t)u(t) = \begin{bmatrix}
0 \\ 1 \\ 0
\end{bmatrix}, \quad x_0 = \begin{bmatrix}
0 \\ 0 \\ 1
\end{bmatrix} \\
x(t) &= \Phi(t,0)x_0 + \int_0^t\Phi(t,\tau)B(\tau)u(\tau)\de\tau \\
&=\begin{bmatrix}
0 \\ 2e^{-2t}-2e^{-t} \\ e^{-2t} \end{bmatrix} + \int_0^t\Phi(t,\tau)\begin{bmatrix}
0 \\ 1 \\ 0
\end{bmatrix}\de\tau \\
&=\begin{bmatrix}
0 \\ 2e^{-2t}-2e^{-t} \\ e^{-2t} \end{bmatrix} + \int_0^t\begin{bmatrix}
0 \\ e^{-t+\tau} \\ 0
\end{bmatrix}\de\tau \\
&=\begin{bmatrix}
0 \\ 2e^{-2t}-2e^{-t} \\ e^{-2t} \end{bmatrix} + \begin{bmatrix}
0 \\ 1-e^{-t} \\ 0 \end{bmatrix} \\
&=\begin{bmatrix}
0 \\ 1+2e^{-2t}-3e^{-t} \\ e^{-2t} \end{bmatrix} \\
y(t) &= \begin{bmatrix}1 & 1 & 0\end{bmatrix}x(t) \\
&=1+2e^{-2t}-3e^{-t}
\end{align*}



\end{document}