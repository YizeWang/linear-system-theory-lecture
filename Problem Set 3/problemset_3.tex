% style settings
\documentclass[a4paper,10.5pt]{article}
\headsep 16mm
\parindent 0mm
\headheight 8mm
\topmargin -20mm
\textwidth 160mm
\textheight 240mm
\baselineskip 7mm
\oddsidemargin -.15in

\usepackage{psfrag}
\usepackage[dvips]{epsfig}
\usepackage{amsmath,amssymb}
\usepackage{theorem}
\usepackage{epsfig}
\usepackage{color}

% import packages
\usepackage{color}
\usepackage{epsfig}
\usepackage{psfrag}
\usepackage{amsmath}
\usepackage{amssymb}
\usepackage{theorem}
\usepackage{multicol}
\usepackage{mathtools}

% define shortcuts
\def\dim{{\mbox{\em dim}}}
\def\ifpart{\textbf{(if) }}
\def\rank{{\mbox{\sc Rank}}}
\def\null{{\mbox{\sc Null}}}
\def\range{{\mbox{\sc Range}}}
\def\domain{{\mbox{\em domain}}}
\def\nullity{{\mbox{\sc Nullity}}}
\def\onlyifpart{\textbf{(only if) }}
\def\solution{\noindent{\it Solution.}}

% define new commands and environment
\newtheorem{lemma}{Lemma}
\newtheorem{proof}{Proof}
\newcommand{\R}{\mathbb{R}}
\newcommand{\N}{\mathbb{N}}
\newcommand{\A}{\mathcal{A}}
\newcommand{\B}{\mathcal{B}}
\newcommand{\defeq}{\vcentcolon=}
\newcommand{\dt}{\, \mathrm{d} t}
\newcommand{\dx}{\, \mathrm{d} x}
\newcommand{\supx}{\sup_{x\neq0}}
\newcommand{\abs}[1]{\left|#1\right|}
\newcommand{\spanop}{\mathop{\rm span}}
\newcommand{\norm}[1]{\left\lVert#1\right\rVert}
\newenvironment{mymat}{\left[\begin{matrix}}{\end{matrix}\right]}

\begin{document}
\thispagestyle{plain}

% title
\vspace*{-1.5cm} {\noindent \rule{15.8cm}{.3mm} \\[.3cm]}
\begin{center} \bf
{\large Linear System Theory \medskip \\
Problem Set 3 \\
Linear Time-Varying Systems, Linear Time-Invariant Systems \medskip
\\ Issue date: Oct. 21, 2019 \\ Due date: Oct. 31, 2019}
\end{center}
\rule{15.8cm}{.3mm} \\[1cm]

% exercise 1
\noindent {\bf Exercise 1. (Linear Time-Varying Systems, $[$50 points in total$]$)}

Let $A_1(t), A_2(t)$, $F(t) \in \R^{n \times n}$ be piecewise continuous matrix functions. Let $\Phi_i$ be the state transition matrix for $\dot{x}(t) = A_i(t)x(t)$, for $i = 1,2$. Consider the matrix differential equation:
\begin{align*}
\dot{X}(t) = A_1(t)X(t) + X(t) A^T_2(t) + F(t), \; X(t_0) = X_0,
\end{align*}
where $X(t)\in \R^{n\times n}$ for any $t\ge t_0$.

\begin{enumerate}
\item $[$20 {\bf points}$]$ Show that this is an affine time-varying system.  (Hint: An affine time-varying system is a system of the form $\dot{x}(t) = A(t)x(t) + b(t)$, where $x(t)$ and $b(t)$ are vectors.)
\item $[$30 {\bf points}$]$ Assume that the solution of the above system can be written as:
\begin{align*}
X(t) = \Phi_1(t,t_0) X_0 \Phi^T_2(t,t_0) + \int_{t_0}^{t}\Phi_1(t,\tau)M(t,\tau)d\tau.
\end{align*}
Express the matrix $M(t,\tau)$ as a function of $\Phi_1(t,\tau)$, $F(t)$, and $\Phi_2(t,\tau)$. (Hint: $\Phi_1(t,\tau)$, $F(t)$, and $\Phi_2(t,\tau)$ may not all appear in the expression of $M(t,\tau)$.)
\end{enumerate}

\clearpage

% exercise 2
\noindent {\bf Exercise 2. (Linear Time-Invariant Systems, $[$50 points in total$]$)}

Consider the following affine system:
\begin{equation*}
  \dot{x}(t) = A\, x(t) + 
  \begin{bmatrix}
0\\ 1\\ 0\end{bmatrix}\, \quad
  y(t) = \begin{bmatrix}1 & 1 & 0\end{bmatrix}\, x(t).
\end{equation*}
where $A \in \R^{3 \times 3}$.
The matrix $A$ has eigenvalues $\lambda_1 = -2$ with multiplicity~2, and $\lambda_2 = -1$ with multiplicity~1.
The eigenvalue $\lambda_1$ has an eigenvector $v_1 = \begin{bmatrix}0\\ 2\\ 1\end{bmatrix}$, and a generalized eigenvector $v_1' = \begin{bmatrix}-1\\ 0\\ 1\end{bmatrix}$.
The eigenvalue $\lambda_2$ has the eigenvector $v_2 = \begin{bmatrix}0\\ 1\\ 0\end{bmatrix}$.
\begin{enumerate}
	\item $[$20 points$]$ Find the matrix $A$.
	\item $[$20 points$]$ Calculate $\exp(A\, t)$.
	\item $[$10 points$]$ Given $x(0)=[0\,\,0\,\,1]^{\top}$, compute $y(t)$.
\end{enumerate}

\end{document}