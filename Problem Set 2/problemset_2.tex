\documentclass[fleqn, 10.5pt, a4paper]{article}

\usepackage{psfrag}
\usepackage[dvips]{epsfig}
\usepackage{amsmath,amssymb}
\usepackage{theorem}
\usepackage{epsfig}
\usepackage{color}

\def\R{{\mathbb R}}
\renewcommand{\Re}{{\mathbb R}}
\def\N{{\mathbb N}}
\def\C{{\mathbb C}}
\def\Z{{\mathbb Z}}
\def\Q{{\mathbb Q}}
\def\C{\mathbb{C}}


\def\I{\mathcal{I}}
\def\A{\mathcal{A}}
\def\B{\mathcal{B}}
\def\U{\mathcal{U}}
\def\V{\mathcal{V}}
\def\W{\mathcal{W}}
\def\F{\mathcal{F}}
\def\Nscr{{\mathcal{N}}}
\def\Mscr{{\mathcal{M}}}

%------------------------------------------------------------
% Theorem like environments
%
\newtheorem{theorem}{Theorem}
\newtheorem{corollary}{Corollary}
\newtheorem{lemma}{Lemma}
\newtheorem{proposition}{Proposition}
%\theoremstyle{plain}
\newtheorem{acknowledgement}{Acknowledgement}
\newtheorem{algorithm}{Algorithm}
\newtheorem{axiom}{Axiom}
\newtheorem{case}{Case}
\newtheorem{claim}{Claim}
\newtheorem{conclusion}{Conclusion}
\newtheorem{condition}{Condition}
\newtheorem{criterion}{Criterion}
\theoremstyle{definition}
\newtheorem{definition}{Definition}
\newtheorem{example}{Example}
\newtheorem{exercise}{Exercise}
\newtheorem{notation}{Notation}
\newtheorem{problem}{Problem}
\newtheorem{remark}{Remark}
\newtheorem{solution}{Solution}
\newtheorem{summary}{Summary}
\numberwithin{equation}{section}
%--------------------------------------------------------

\def\bmat{\left[ \begin{array}}
\def\emat{\end{array} \right]}

\newcommand{\spanop}{\mathop{\rm span}}
\def\bmat{\left[ \begin{array}}
\def\emat{\end{array} \right]}

\headheight 8mm
\headsep 16mm
\topmargin  -20mm
\oddsidemargin -.15in
\textwidth 160mm
\textheight 240mm
\baselineskip 7mm
\parindent 0mm
%\pagestyle{headings}
%\markright{EIST: Paper 3 Examples Paper 1 }

\def\domain{{\mbox{\em domain}}}
\def\dim{{\mbox{\em dim}}}
\def\range{{\mbox{\sc Range}}}
\def\null{{\mbox{\sc Null}}}
\def\nullity{{\mbox{\sc Nullity}}}
\def\rank{{\mbox{\sc Rank}}}

%\def\nulls{\mbox{\sc Null}}
%def\range{\mbox{\sc Range}}
%\newcommand{\spn}{\mbox{\sc Span}}
%\def\rank{\mbox{\sc Rank}



\def\ifpart{\textbf{(if) }}
\def\onlyifpart{\textbf{(only if) }}
\def\solution{\noindent{\it Solution.} }

\newcommand{\dx}{\, \mathrm{d} x}
\newcommand{\supx}{\sup_{x\neq0}}
\newcommand{\norm}[1]{\left\lVert#1\right\rVert}

\begin{document}
\thispagestyle{plain}

%%%%%%%%%%%%%%%%%%%%%%%%%%%%%%%%%%%%%%%%%%%%%%%%%%%%%%%%%%%%%%%%%%%%%%
\vspace*{-1.5cm}
{\noindent \rule{15.8cm}{.3mm} \\[.3cm]}
\begin{center} \bf
{\large Linear System Theory \medskip
\\
Problem Set 2 \\
Normed Spaces, ODEs, and Linear Time-Varying Systems\medskip
\\ Issue date: October 7, 2019
\\ Due date: October 21, 2019}
\end{center}
\rule{15.8cm}{.3mm} \\[0cm]
%%%%%%%%%%%%%%%%%%%%%%%%%%%%%%%%%%%%%%%%%%%%%%%%%%%%%%%%%%%%%%%%%%%%%%

%==========================================================================================================
% 2016,2
\noindent
{\bf Exercise 1. (Norms, {\bf$[45\, \text{points in total}]$})}
\noindent
\begin{enumerate}
\item {\bf$[15\, \text{points}]$}
Let $C([t_0, t_1], \R^n)$ be the space of all continuous functions from $[t_0, t_1]$ to $\R^n$. Prove that for $f \, \in \, C([t_0, t_1], \R^n)$,
 $\left\lVert f \right\rVert_\infty := \max\limits_{t \in [t_0, t_1]} \|f(t)\|_p \, $ satisfies the axioms of the norm, where $\|x\|_p$ is the $p$-norm of $x \in \R^n$.

\item {\bf$[10\, \text{points}]$}
Given a matrix $A \in \R^{m \times n}$, verify that the induced matrix norms $\|A\|_2 \, ,\|A\|_\infty$ are equivalent, by showing that they satisfy the following inequalities: 
\begin{displaymath}
\dfrac{1}{\sqrt{n}} \|A\|_\infty \leq \|A\|_2 \leq \sqrt{m} \|A\|_\infty.
\end{displaymath}

{\em Hint:} The induced $p$-norm of a matrix $A$ is given by:
\begin{displaymath}
\|A\|_p=\sup\limits_{\|x\|_p \neq 0} \dfrac{\|Ax\|_p}{\|x\|_p}
\end{displaymath}

\item {\bf$[20\, \text{points}]$}
Consider a set of functions $f_n$ in $C([0,1], \R)$ defined as:
\begin{displaymath}
f_n:[0,1] \rightarrow \R \quad\text{s.t.}\quad
f_n(x) = 
\begin{cases}
n -n^2x & 0\leq x\leq \frac{1}{n}\\
0 & \text{elsewhere}
\end{cases}
\end{displaymath}
for $n\in\N$. Compute the 1-norm, 2-norm and $\infty$-norm of the functions for each n, as defined below:
\begin{displaymath}
\left\lVert f \right\rVert_1 := \int_0^1 \left\lvert f(x) \right\rvert dx\, , \quad
\left\lVert f \right\rVert_2 := \sqrt{\int_0^1 \left\lvert f(x) \right\rvert^2 dx}\, , \quad
\left\lVert f \right\rVert_\infty := \max\limits_{t \in [0, 1]} |f(t)|.
\end{displaymath}
Based on your computations, what can you say about equivalence of these norms?  
\end{enumerate} 

\noindent \textbf {Solution 1.1} \\
\begin{itemize}
	\item $\forall f, g \in C([t_0, t_1], \R^n), \|f+g\|_\infty \leq \|f\|_\infty + \|g\|_\infty$
	\begin{align*}
		\forall f, g \in C([t_0, t_1], \R^n), \|f+g\|_\infty &= \max_{t \in [t_0, t_1]} \| f + g\|_p \\
		&\leq  \max_{t \in [t_0, t_1]} \{\| f \|_p + \|g\|_p\} \\
		&\leq \max_{t \in [t_0, t_1]} \| f \|_p + \max_{t \in [t_0, t_1]} \|g\|_p \\
		&= \|f\|_\infty + \|g\|_\infty
	\end{align*}
	\item $\forall f \in C([t_0, t_1], \R^n), \forall a \in \R, \|af\| = a\|f\|$
	\begin{align*}
		\forall f \in C([t_0, t_1], \R^n), \forall a \in \R, \|af\| &= \max_{t \in [t_0, t_1]} \| af \|_p \\
		&= \max_{t \in [t_0, t_1]} a \| f \|_p \\
		&= a \max_{t \in [t_0, t_1]} \| f \|_p \\
		&= a\| f \|_\infty \\
	\end{align*}
	\item $\|f\|=0 \Leftrightarrow f = 0$ \\
	$\Rightarrow: \|f\|=0\Rightarrow\max_{t \in [t_0, t_1]} \| f \|_p = 0$, $\|f\|_p \geq 0 \Rightarrow \|f\|_p = 0, \forall t \in [t_0, t_1] \Rightarrow f = 0$ \\
	$\Leftarrow: f = 0 \Rightarrow \|f\|_p = 0, \forall t \in [t_0, t_1] \Rightarrow \max_{t \in [t_0, t_1]} \| f \|_p = 0 \Rightarrow \|f\|=0$
\end{itemize}

\noindent \textbf {Solution 1.2} \\
Define $x_{max}$ and $\left(Ax\right)_{max}$ to simplify further notation:
\begin{align*}
	x_{max} &= \max_{i\in\{1,\dots,n\}} x_i \\
	\left(Ax\right)_{max} &= \max_{j\in\{1,\dots,m\}}\left(Ax\right)_j
\end{align*}
Then, start with $\|A\|_2^2$:
\begin{align*}
\|A\|_2^2&=\left(\supx\frac{\|Ax\|_2}{\|x\|_2}\right)^2=\supx\frac{\|Ax\|_2^2}{\|x\|_2^2} = \supx \frac{\sum_{i=1}^{m} \left(Ax\right)_i^2}{\sum_{i=1}^{n} x_i^2} \geq \supx \frac{\sum_{i=1}^{m} \left(Ax\right)_i^2}{\sum_{i=1}^{n} x_{max}^2} \\
&=\supx \frac{1}{n}\frac{\sum_{i=1}^{m} \left(Ax\right)_i^2}{x_{max}^2} \geq \supx \frac{1}{n}\frac{\left(Ax\right)_{max}^2}{x_{max}^2} = \supx \frac{1}{n} \left(\frac{\left(Ax\right)_{max}}{x_{max}}\right)^2 \\
&= \frac{1}{n}  \left(\supx\frac{\left(Ax\right)_{max}}{x_{max}}\right)^2 =\frac{1}{n} \left(\supx \frac{\|Ax\|_\infty}{\|x\|_\infty}\right)^2= \frac{1}{n}\|Ax\|_\infty \\
&\Rightarrow \|A\|_2 \geq \frac{1}{\sqrt{n}}\|Ax\|_\infty
\end{align*}
\begin{align*}
\|A\|_2^2&=\left(\supx\frac{\|Ax\|_2}{\|x\|_2}\right)^2=\supx\frac{\|Ax\|_2^2}{\|x\|_2^2} = \supx \frac{\sum_{i=1}^{m} \left(Ax\right)_{i}^2}{\sum_{i=1}^{n} x_i^2} \leq \supx \frac{\sum_{i=1}^{m} \left(Ax\right)_{max}^2}{\sum_{i=1}^{n} x_{i}^2} \\&= \supx \frac{m \left(Ax\right)_{max}^2}{\sum_{i=1}^{n} x_{i}^2} = m\supx \frac{\left(Ax\right)_{max}^2}{\sum_{i=1}^{n} x_{i}^2} \leq m\supx \frac{\left(Ax\right)_{max}^2}{x_{max}^2} = m\supx \left(\frac{\left(Ax\right)_{max}}{x_{max}}\right)^2 \\
&=m\left(\supx \frac{\left(Ax\right)_{max}}{x_{max}}\right)^2 = m\left(\supx \frac{\|Ax\|_\infty}{\|x\|_\infty}\right)^2 = m\|Ax\|_\infty \\
&\Rightarrow \|A\|_2 \leq \sqrt{m}\|Ax\|_\infty
\end{align*}

\noindent \textbf {Solution 1.3} \\
Although $n\in\N$ can be any non-negative integer, we additionally require $n>0$ to make sure the piece-wise function $f_0(x)$ is well defined, otherwise $\frac{1}{0}$ does not make sense.
\begin{align*}
\forall x \in \left[0,\frac{1}{n}\right], \forall n \in \N, n - n^2 x \geq 0 \Rightarrow f_n (x) \geq 0, \forall x \in [0,1] \Rightarrow \lvert f_n(x) \rvert = f_n(x), \forall x \in [0,1] \\
\end{align*}

\begin{itemize}
\item $\|f\|_1$
\begin{align*}
	\|f\|_1 = \int_0^1 \lvert f_n(x)\rvert \dx=\int_0^1f_n(x)\dx = \int_0^{\frac{1}{n}}\left(n-n^2x\right) \dx = nx-\frac{1}{2}n^2x^2 \bigg\rvert_0^{\frac{1}{n}} = \frac{1}{2}
\end{align*}
\item $\|f\|_2$
\begin{align*}
\|f\|_2^2&=\int_0^1 \lvert f_n(x)\rvert ^2 \dx = \int_0^{\frac{1}{n}}\left(n-n^2x\right)^2\dx \\
&=\int_0^{\frac{1}{n}}\left(n^4x^2-2n^3x+n^2\right) \dx = \frac{1}{3}n^4x^3-n^3 x^2 + n^2 x \bigg\rvert_0^{\frac{1}{n}}=\frac{1}{3}n \\
\|f\|_2 &= \sqrt{\frac{1}{3}n}
\end{align*}
\item $\|f\|_\infty$
\begin{align*}
\|f\|_\infty = \max_{t \in [0,1]} \lvert f_n(t) \rvert = \max_{t\in \left[0,1/n\right]} n-n^2t=n
\end{align*}
\item Equivalence of $\|f_n\|_1$ and $\|f_n\|_\infty$
\begin{align*}
\exists m_u = 4n \geq m_l = n \geq 0, \forall f_n \in C([0,1], \R), m_l\|f_n\|_1 = \frac{1}{2}n \leq n = \|f_n\|_\infty \leq 2n = m_u\|f_n\|_1
\end{align*}
\item Equivalence of $\|f_n\|_2$ and $\|f_n\|_\infty$
\begin{align*}
\exists m_u = \sqrt{27n} \geq m_l = \sqrt{\frac{1}{3}n} \geq 0, \forall f_n \in C([0,1], \R), m_l\|f_n\|_2 = \frac{1}{3}n \leq n = \|f_n\|_\infty \leq 3n = m_u\|f_n\|_2
\end{align*}
Therefore,  $\|f_n\|_1$, $\|f_n\|_2$ and $\|f_n\|_\infty$ are equivalent.
\end{itemize}







%%%%%%%%%%%%%%%%%%%%%%%%%%%%%%%%%%%%%%%%%%%%%%%%%%%%%%%%%%%%%%%%%%%%%%
\clearpage
%2017, 2
{\bf Exercise 2. (Banach fixed point theorem  {\bf$[25\, \text{points in total}]$})}
\begin{enumerate}
\item {\bf$[20\, \text{points}]$} Let $\left(X, \left\lVert\cdot\right\rVert\right)$ be a Banach space, and $f:X\rightarrow X$. Assume that there exists $\alpha\in[0,\, 1)$ such that, for all $x,y\in X$,
\begin{displaymath}
\left\lVert f(x) - f(y) \right\rVert
\leq
\alpha
\left\lVert x - y \right\rVert.
\end{displaymath}
Show that there exists a unique point $\bar{x}$ such that $f(\bar{x}) = \bar{x}$.

\noindent {\em Hint:} 
Given an arbitrary initial point $x$,
consider the sequence of iterates 
$f^{[n]}(x) = f(f^{[n-1]}(x))$, where the first iterate is given by $f^{[0]}(x)=x$. You can start by showing that this sequence is Cauchy.

\item {\bf$[5\, \text{points}]$} Now assume $f$ is a linear map. Given the condition in the first part of Exercise 2, what can you conclude about the induced norm of $f$?

\end{enumerate}

\noindent \textbf {Solution 2.1} \\
\begin{align*}
	f^{[0]}(x) &= x \\
	f^{[1]}(x) &= f\left(f^{[0]}(x)\right) = f(x) \\
	f^{[2]}(x) &= f\left(f^{[1]}(x)\right) = f(f(x)) \\
			   & \;\; \vdots \\
	f^{[n]}(x) &= f\left(f^{[n-1]}(x)\right) = f(\cdots f(f(x)))		  
\end{align*}
$\forall p\geq q\geq 0, p,q\in N:$
\begin{align*}
	\|f^{[p]}(x)-f^{[q]}(x)\| &= \norm{f\left(f^{[p-1]}(x)\right)-f\left(f^{[q-1]}(x)\right)} \\
	&\leq \alpha \norm{f^{[p-1]}(x)-f^{[q-1]}(x)} \\
	& \; \; \vdots \\
	&\leq \alpha^{q} \norm{f^{[p-q]}(x)-f^{[0]}(x)} \\
	&= \alpha^q \norm{f^{[p-q]}(x)-x}
\end{align*}
$\forall k \in N:$
\begin{align*}
	\norm{f^{[k]}(x)-x} &= \norm{f^{[k]}(x)-f(x)+f(x)-x} \\
	&\leq \norm{f^{[k]}(x)-f(x)}+\norm{f(x)-x} \\
	&\leq \alpha\norm{f^{[k-1]}(x)-x}+\norm{f(x)-x} \\
	&\leq \alpha^2\norm{f^{[k-2]}(x)-x}+\left(1+\alpha\right)\norm{f(x)-x} \\
	& \;\; \vdots \\
	&\leq \alpha^{k-1}\norm{f^{[1]}(x)-x} + \left(1+\alpha+\cdots+\alpha^{k-2}\right)\norm{f(x)-x} \\
	&\leq \left(1+\alpha+\cdots+\alpha^{k-2}+\alpha^{k-1}\right)\norm{f(x)-x} \\
	&=\frac{1-\alpha^k}{1-\alpha}\norm{f(x)-x}
\end{align*}
$\forall \epsilon>0,\exists N = \bigg\lceil \log_\alpha\left(\frac{\epsilon(1-\alpha)}{\norm{f(x)-x}}+\alpha^m\right) \bigg\rceil \in \N \ (\lceil t \rceil \text{ means the smallest integer greater than } t),\forall m \geq N,$
\begin{align*}
\norm{f^{[m]}(x)-f^{[N]}(x)} 
&\leq \alpha^N \norm{f^{[m-N]}(x)-x} \\
&\leq \alpha^N \cdot \frac{1-\alpha^{m-N}}{1-\alpha} \cdot \norm{f(x)-x} \\
&= \frac{\alpha^N-\alpha^{m}}{1-\alpha} \norm{f(x)-x} \\
N&>\log_\alpha\left(\frac{\epsilon(1-\alpha)}{\norm{f(x)-x}}+\alpha^m\right) \\
\Rightarrow \norm{f^{[m]}(x)-f^{[N]}(x)} &< \epsilon
\end{align*}
Therefore, $\{f^{[i]}\}_{i=0}^{\infty}$ is a Cauchy sequence.
Note that given an arbitrary initial point $x$, we can compute the value of $\|f(x)-x\|$ so we treat $\|f(x)-x\|$ as a known constant in the proof above. Because $\left(X, \left\lVert\cdot\right\rVert\right)$ is a Banach space, $\{f^{[i]}\}_{i=0}^{\infty}$ converges to a point $f^*$.
\begin{align*}
	f^* &=\lim_{n\to\infty}f^{[n]}(x) = \lim_{n\to\infty}f\left(f^{[n-1]}(x)\right) = f\left(\lim_{n\to\infty}f^{[n-1]}(x)\right) = f(f^*) 
\end{align*}
Therefore, there exist $\bar{x}=f^*$ such that $f(\bar{x}) = \bar{x}$. Next, we prove uniqueness: For the sake of contradiction, assume there exist $\bar{x}_1,\bar{x}_2 \in X, \bar{x}_1 \neq \bar{x}_2$ such that $f(\bar{x}_1)=\bar{x}_1$ and $f(\bar{x}_2)=\bar{x}_2$.
\begin{align*}
	\norm{f(\bar{x}_2)-f(\bar{x}_1)} =\norm{\bar{x}_2-\bar{x}_1} \leq \alpha\norm{\bar{x}_2-\bar{x}_1}	
\end{align*}
Since $\alpha \in [0,1)$, $\alpha \neq 1$. The above inequality never holds, which leads to contradiction, so $\bar{x}$ is unique.
\noindent \textbf {Solution 2.2} \\
Let $F$ be the representation of linear map $f(\cdot)$. $\forall x,y \in X$:
\begin{align*}
\norm{f(x)-f(y)} = \norm{Fx-Fy} = \norm{F(x-y)} \leq \alpha \norm{x-y} \Rightarrow \frac{\norm{F(x-y)}}{\norm{x-y}} &\leq \alpha
\end{align*}
$\forall u \in X, \exists x,y \in X$ such that $u = x-y$.
\begin{align*}
\norm{F} = \sup_{u\neq 0}\frac{\norm{Fu}}{\norm{u}} \leq \sup_{u\neq 0} \alpha = \alpha
\end{align*}
Therefore, the induced norm of $f$ is bounded by $\alpha$.

%%%%%%%%%%%%%%%%%%%%%%%%%%%%%%%%%%%%%%%%%%%%%%%%%%%%%%%%%%%%%%%%%%%%%%
\bigskip

\noindent
% 2017, 2
{\bf Exercise 3. (Ordinary differential equations {\bf$[30\, \text{points in total}]$})}
\begin{enumerate}
	\item {\bf$[12\, \text{points}]$} Consider the following ordinary differential equation (ODE)
	\begin{displaymath}
	\begin{bmatrix}\dot{x}_1(t)\\ \dot{x}_2(t)\end{bmatrix}= \begin{bmatrix}
	-x_1(t)+e^t\, \cos(x_1(t)-x_2(t)) \\ -x_2(t)+\sin(x_1(t)-x_2(t))
	\end{bmatrix},
	\end{displaymath}
	where $x_i(t) \in \R,\, \forall i$. Prove or disprove the following statements:
	\begin{enumerate}
		\item This system is globally Lipschitz,
		\item This system admits a unique solution.
		\\
		{\em Hint:} You may assume that functions with bounded derivatives are Lipschitz.
	\end{enumerate}
\item {\bf$[18\, \text{points}]$} Consider the following ordinary differential equation (ODE)
\begin{displaymath}
\begin{bmatrix}\dot{x}_1(t)\\ \dot{x}_2(t)\end{bmatrix}= \begin{bmatrix}
-3 \sin(t)\, x_1(t) + x_1(t)\, x_2(t) \\ -2\, x_2(t)
\end{bmatrix},
\end{displaymath}
where $x_i(t) \in \R,\, \forall i$. Prove or disprove the following statements:
\begin{enumerate}
	\item This system is globally Lipschitz,
	\item This system admits a unique solution.
\end{enumerate}
\end{enumerate}

%%%%%%%%%%%%%%%%%%%%%%%%%%%%%%%%%%%%%%%%%%%%%%%%%%%%%%%%%%%%%%%%%%%%%%


%%%%%%%%%%%%%%%%%%%%%%%%%%%%%%%%%%%%%%%%%%%%%%%%%%%%%%%%%%%%%%%%%%%%%%

\end{document}
