\documentclass[a4paper,12pt]{book}

% import packages
\usepackage{amsmath,amsthm,amssymb}
\usepackage{graphicx}
\usepackage{mathrsfs}
\usepackage[utf8]{inputenc}
\usepackage{pgf,tikz,pgfplots}
\usetikzlibrary{arrows}

% define new counters (reset upon each chapter)
\newcounter{fact}[chapter]
\newcounter{problem}[chapter]
\newcounter{exercise}[chapter]
\newcounter{tutorial}[chapter]

% define shortcuts
\newcommand{\R}{\mathbb{R}}
\newcommand{\N}{\mathbb{N}}
\newcommand{\I}{\mathbb{I}}
\newcommand{\X}{\mathcal{X}}
\newcommand{\Y}{\mathcal{Y}}
\newcommand{\range}[1]{{\mbox{\sc{Range}}}(#1)}
\newcommand*{\QED}{\par\hfill\ensuremath{\blacksquare}}

% define new commands
\newcommand{\fact}{%
	\subsubsection*{\refstepcounter{Fact}Fact \thechapter.\theproblem}}
\newcommand{\problem}{%
	\subsubsection*{\refstepcounter{problem}Problem \thechapter.\theproblem}}
\newcommand{\exercise}{%
	\subsubsection*{\refstepcounter{exercise}Exercise \thechapter.\theexercise}}
\newcommand{\tutorial}{%
	\subsubsection*{\refstepcounter{tutorial}Tutorial Exercise \thechapter.\thetutorial}}
\newcommand{\tutorialsolution}{%
	\subsubsection*{Tutorial Exercise Solution \thechapter.\thetutorial}}

\begin{document}

% front page
\author{Yize Wang}
\title{Linear System Theory\\Solution Set}
\date{\today}

% table of contents
\frontmatter
\maketitle
\tableofcontents

% main contents
\mainmatter
\chapter{Introduction}
\label{sec:introduction}

\exercise
\begin{center}
	\begin{tikzpicture}
	\draw (0,0) rectangle (8,5.5);
	\draw (4,2.5) ellipse (3.5 and 2);
	\draw (4,1.75) ellipse (2 and 1);
	\draw (0,5.5) node[anchor = north west]{People};
	\draw (4,3.5) node[anchor = center]{non-Greek};
	\draw (4,1.75) node[anchor = center]{non-European};
	\end{tikzpicture}
\end{center}

\exercise
\begin{center}
	\begin{tikzpicture}
	\draw (0,0) rectangle (12,6);
	\draw (4,3) ellipse (3.5 and 2.5);
	\draw (8,3) ellipse (3.5 and 2.5);
	\draw (4,3) ellipse (3 and 1);
	\draw (8,3) ellipse (3 and 1);
	\draw (0,6) node[anchor = north west]{People};
	\draw (4,5) node[anchor = center]{European};
	\draw (8,5) node[anchor = center]{Asian};
	\draw (3,3) node[anchor = center]{Greek};
	\draw (9,3) node[anchor = center]{Chinese};
	\end{tikzpicture}
\end{center}

\exercise
statement $p$: $\sqrt{2} \in \R$ \\
statement $q$: $\sqrt{2}$ is not rational \\
statement $r$: $m$ and $n$ have no common divisor

\exercise
$f \circ (g \circ h)(w) = f \circ g(h(w)) = f(g(h(w))) = (f \circ g)(h(w))=(f \circ g) \circ h(w)$ \QED

\exercise
identity map definition: $\forall x \in \X, 1_X(x) = x$ \\
injective: $1_X(x_1) = 1_X(x_2) \Rightarrow x_1 = x_2$ \\
surjective: $\forall y \in \X, \exists x = y \in \X, 1_X(x) = y$ \\
injective $\wedge$ surjective $\Rightarrow$ bijective \QED

\problem
Assume, for the sake of contradiction, $\exists n \in \N$, such that $n$ is both odd and even. According to the definitions of even and odd numbers, we know that $\exists p \in \N, n = 2p+1$ and $\exists q \in \N, n = 2q$. Thus, $2p+1 = 2q$ and hence $q = p + 0.5$. Since $p \in \N$, $q = p + 0.5 \notin \N$, which leads to the contradiction with $q \in \N$. Therefore, $n$ cannot be both odd and even. \QED \\

\problem
\begin{enumerate}
\item $f$ has a left inverse if and only if it is injective:
\begin{itemize}
	\item Suppose $f$ has a left inverse $g_L$, \\
	$g_L:\Y \rightarrow \X$ such that $\forall x\in \X, g_L \circ f(x) = x$ \\
	$f(x_1)=f(x_2) \Rightarrow g_L(f(x_1)) = g_L(f(x_2)) \Rightarrow x_1 = x_2 \Rightarrow$ injective
	\item Suppose $f$ is injective, $\forall y \in \range{f}$, $\exists ! x \in \X, f(x) = y$. \\
	Construct $g(y)_L = \begin{cases}
	x & y \in \range{f} \\
	0 & otherwise
	\end{cases}$ \\
	Then $g_L(y)$ defines the left inverse function of $f$.
\end{itemize}

\item $f$ has a right inverse if and only if it is surjective:
\begin{itemize}
	\item Suppose $f$ has a right inverse $g_R$, $\forall y \in \Y, f \circ g(y) = y$. \\
	$\forall y \in \Y, \exists x = g(y) \in \X, \text{such that } f(x) = y \Rightarrow$ surjective
	\item Suppose $f$ is surjective, $\forall y \in \Y, \exists x \in \X, f(x) = y$. \\
	Let $g_R(y) = x$. If multiple $x$ exist, choose one of them.
	$f(g_R(y)) = f(x) = y = 1_\Y(y)$ \\
	Therefore $g_R(y)$ defines the right inverse function of $f$.
\end{itemize}

\item $f$ is invertible if and only if it is bijective:
\begin{itemize}
	\item Suppose $f$ is invertible, there exist left and right inverse of $f$. Then $f$ is both injective and surjective, which implies $f$ is bijective.
	\item Suppose $f$ is bijective, we only need to prove $g = g_L = g_R$. \\
	$g_L(y) = g_L \circ 1_\Y(y) = g_L(1_\Y(y)) = g_L(f\circ g_R(y)) = (g_L \circ f) \circ g_R (y) = 1_\X \circ g_R (y) = g_R(y) $
\end{itemize}
\end{enumerate} \QED
\chapter{Introduction to Algebra}

\exercise

\exercise
\chapter{Introduction to Analysis}
\chapter{Time Varying Linear Systems: Solutions}

\tutorial
Given the system dynamics $\dot{x}(t) = A(t)x(t)$ and let $\Phi(t,t_0)$ be the transition matrix. Show that 
\begin{align*}
	\frac{\partial}{\partial t}\Phi(t_0,t)=-\Phi(t_0,t)A(t) \; .
\end{align*}
\tutorialsolution
\begin{align*}
\Phi(t,t_0)\Phi(t_0,t) &= \I \\
\frac{\partial}{\partial t}\Phi(t,t_0)\Phi(t_0,t) + \Phi(t,t_0)\frac{\partial}{\partial t}\Phi(t_0,t) &= 0 \\
A\Phi(t,t_0)\Phi(t_0,t)+\Phi(t,t_0)\frac{\partial}{\partial t}\Phi(t_0,t) &= 0 \\
\Phi(t,t_0)\frac{\partial}{\partial t}\Phi(t_0,t) &= -A \\
\frac{\partial}{\partial t}\Phi(t_0,t) &= -\Phi(t_0,t)A(t)
\end{align*}

\clearpage

\tutorial
Given the system dynamics $\dot{x}(t) = A(t)x(t)$ and let $\Phi(t,t_0)$ be the transition matrix. Prove that if $\X_0 \subseteq \R^n$ is convex, then $\forall t \geq 0$, the set $\X(t) = \{s(t,t_0,x_0,0):x \in \X_0\}$ is convex. \\
\tutorialsolution
\begin{flalign*}
	&\forall s_1,s_2 \in \X, \exists x_0^1, x_0^2 \in \X_0, \text{such that } s_1 = s(t,t_0,x_0^1,0), s_2 = s(t,t_0,x_0^2,0) &\\
	&\forall \lambda \in [0,1], \lambda s_1 + (1-\lambda)s_2 = \lambda s(t,t_0,x_0^1,0) + (1-\lambda)s(t,t_0,x_0^2,0) &\\
	&\phantom{\forall \lambda \in [0,1], \lambda s_1 + (1-\lambda)s_2} = s(t,t_0,\lambda x_0^1 + (1-\lambda)x_0^2,0) &\\
	&\X_0 \text{ is convex} \Rightarrow \lambda x_0^1 + (1-\lambda)x_0^2 \in \X_0 \Rightarrow \lambda s_1 + (1-\lambda)s_2 \in \X &
\end{flalign*}
Therefore, $\X(t)$ is a convex set.
\chapter{First Meeting Discussion}

\subsubsection*{Discussion}
\begin{itemize}
\item For the first phase, we meet several times to modify the style. For the next phase, I discuss PhD students to make sure solutions are correct and concise.
\item For version 1.0, I will simply copy the exercise statements. But many of them heavily depend on the context. After finishing the main part, I will come back to this. Or we can simply include only solution and exercise index.
\item It is better to have a label for each exercise so that you can freely add or delete them.
\item In the homework, commands of \LaTeX and \TeX are mixed.
\item Maybe add a Theorem collection or index.
\item In Page 71, I think it's better to write $B$ instead of $B(\tau)$ because it is not the standard form of convolution, although they lead to the same result.
\item In Page 71, I think you mean ``In some cases this can be done directly from the \textbf{finite} series''.
\item In Page 74, may be use symmetric instead of diagnoal?
\item Is Definitioin 5.1 rigorous?
\end{itemize}

\subsubsection*{Questions}
\begin{itemize}
\item Am I going to have a position in the office?
\item You corrected typos with an additional document instead of fixing them in the lecture notes. Is there something wrong with the source file?
\item What is the difference between ``linear in $u$ and $x$'' and ``linear jointly in $u$ and $x$''?
\item When does the system diverge exponentially?
\item Is the script your work or student's work?
\item What does $\{u_j\}_{j=1}^n \xrightarrow[]{A \in F^{m \times n}} \{v_i\}_{i=1}^m$ mean?
\end{itemize}

% reference and appendix
\backmatter

\end{document}