% style settings
\documentclass[a4paper,10.5pt]{article}
\headsep 16mm
\parindent 0mm
\headheight 8mm
\topmargin -20mm
\textwidth 160mm
\textheight 240mm
\baselineskip 7mm
\oddsidemargin -.15in

% import packages
\usepackage{color}
\usepackage{epsfig}
\usepackage{psfrag}
\usepackage{amsmath}
\usepackage{amssymb}
\usepackage{theorem}
\usepackage{multicol}
\usepackage{mathtools}

% define shortcuts
\def\dim{{\mbox{\em dim}}}
\def\ifpart{\textbf{(if) }}
\def\rank{{\mbox{\sc Rank}}}
\def\nullspace{{\mbox{\sc Null}}}
\def\range{{\mbox{\sc Range}}}
\def\domain{{\mbox{\em domain}}}
\def\nullity{{\mbox{\sc Nullity}}}
\def\onlyifpart{\textbf{(only if) }}
\def\solution{\noindent{\it Solution.}}

% define new commands and environment
\newtheorem{lemma}{Lemma}
\newtheorem{proof}{Proof}
\newcommand{\de}{\mathrm{d}}
\newcommand{\R}{\mathbb{R}}
\newcommand{\N}{\mathbb{N}}
\newcommand{\I}{\mathbb{I}}
\newcommand{\A}{\mathcal{A}}
\newcommand{\B}{\mathcal{B}}
\newcommand{\defeq}{\vcentcolon=}
\newcommand{\dt}{\, \mathrm{d} t}
\newcommand{\dx}{\, \mathrm{d} x}
\newcommand{\supx}{\sup_{x\neq0}}
\newcommand{\abs}[1]{\left|#1\right|}
\newcommand{\spanop}{\mathop{\rm span}}
\newcommand{\norm}[1]{\left\lVert#1\right\rVert}
\newenvironment{mymat}{\left[\begin{matrix}}{\end{matrix}\right]}

\begin{document}
\thispagestyle{plain}

% title
\vspace*{-1.5cm}{\noindent \rule{15.8cm}{.3mm} \\[.3cm]}
\begin{center} \bf
{\large Linear System Theory \medskip \\
Problem Set 1 \\
Linear Spaces, Linear Maps, and Representations\medskip \\
Issue date: September 19, 2019 \\ Due date: October 7, 2019}
\end{center}
\rule{15.8cm}{.3mm} \\[0cm]

\noindent{\bf Exercise 1. (Linear spaces {\bf$[40\, \text{points}]$})}
\begin{enumerate}
	\item {\bf$[18\, \text{points}]$}
	Let $S$ be a set, and $F = \{f: S\rightarrow \R_+ \}$ be the space of functions from $S$ to
	the (strictly) positive reals. Let the operations $\oplus:F\times F \rightarrow F,\ \odot:\R\times F\rightarrow F$ be defined as
	follows:
	\begin{align*}
	[f_1 \oplus f_2] (x) &= f_1(x)f_2(x) \quad\quad \forall f_1,f_2 \in F, \forall x\in S \\
	[\alpha \odot f] (x) &= f(x)^\alpha \quad\quad \forall \alpha\in\mathbb{R}, \forall f \in F, \forall x\in S 
	\end{align*}
	\begin{itemize}
		\item Show that $(F, \R, \oplus,\odot)$ is a linear space. 
		\item Identify the zero-vector.
	\end{itemize}
	
	\item {\bf$[12\, \text{points}]$}
	Let $S=\{a,b\}$, and let
	\begin{align*}
	&f_1(a) = 2,\ f_1(b) = 1 \\
	&f_2(a) = 1,\ f_2(b) = 3 \\ 
	&f_3(a) = 4,\ f_3(b) = 1 
	\end{align*}
	Show that $\{f_1, f_2\}$ are linearly independent and that $\{f_1, f_3\}$ are linearly dependent.
	
	\item {\bf$[10\, \text{points}]$}
	Let $\varphi:F\rightarrow F$ be defined as follows:
	\begin{align*}
	[\varphi(f)] (x) = \sqrt{f(x)} &\quad\quad \forall f\in F, \forall x\in S 
	\end{align*}
	Show that $\varphi$ is a linear map over the space $F$ on $(F, \mathbb{R}, \oplus, \odot)$.
	
\end{enumerate}

\clearpage

\noindent {\textbf {Solution 1.1}}
\begin{itemize}
	\item \textbf{vector addition}
	\begin{itemize}
		\item \textbf{associative}: \\
		$\forall f_1, f_2, f_3 \in F, f_1 \oplus (f_2 \oplus f_3) = f_1 \oplus (f_2 f_3) = f_1 (f_2 f_3) = f_1 f_2 f_3 = (f_1 f_2) f_3 = (f_1 \oplus f_2 ) \oplus f_3$
		\item \textbf{commutative}: \\
		$\forall f_1, f_2 \in F, f_1 \oplus f_2 = f_1 f_2 = f_2 f_1 = f_2 \oplus f_1$
		\item \textbf{identity}: \\
		Define $\theta(x) = 1, \forall x \in S$ \\
		$F = \{f: S\rightarrow \R_+ \} \Rightarrow \theta(x) \in F$ \\
		$\forall f \in F, f \oplus \theta = f \cdot 1 = f$
		\item \textbf{inverse}: \\
		$\forall f \in F, f(x) \in \R_+, f(x) > 0, f^{-}(x) = \frac{1}{f(x)} > 0, f^{-} \in F, f \oplus f^{-} (x) = f(x) \cdot \frac{1}{f(x)} = 1$
	\end{itemize}
	\item \bf{scalar multiplication}
	\begin{itemize}
		\item associative: \\
		$\forall a, b \in \R, \forall f \in F, a \odot (b \odot f) = a \odot f^b = (f^b)^a = f^{ab} = (a \cdot b) \odot f$
		\item identity: \\
		$\forall f \in F, 1 \odot f = f^1 = f$
	\end{itemize}
	\item \bf{distributive scalar multiplication}
	\begin{itemize}
		\item $\forall a, b \in \R, \forall f \in F, (a + b) \odot f = f^{a+b} = f^a f^b = (f^a) (f^b) = (a \odot f) \oplus (b \odot f)$
		\item $\forall a \in \R, \forall f_1, f_2 \in F, a \odot (f_1 \oplus f_2) = a \odot (f_1 f_2) = \left(f_1 f_2\right)^a = f_1^a f_2^a = (a \odot f_1) \oplus (a \odot f_2)$
	\end{itemize}
\end{itemize}
Additionally, the addition and multiplication are closed in space $F$. Therefore, $(F, \R, \oplus,\odot)$ is a linear space and the zero-vector is $1$. \\

\noindent {\bf Solution 1.2}

\begin{itemize}
	\item $\{f_1,f_2\}$ \\
	Assume, for the sake of contradiction, $\{f_1,f_2\}$ are linearly dependent such that \\
	$\exists c_1 ,c_2 \in \R \text{ which are not both 0 such that} \left( c_1 \odot f_1(x) \right) \oplus \left( c_2 \odot f_2(x) \right) = 1$ \\
	$x = a$: $\left( c_1 \odot 2 \right) \oplus \left( c_2 \odot 1 \right) = 1 \Rightarrow 2^{c_1} \cdot 1^{c_2} = 2^{c_1} = 1 \Rightarrow c_1 = 0$ \\
	$x = b$: $\left( 0 \odot 1 \right) \oplus \left( c_2 \odot 3 \right) = 1 \Rightarrow 2^{0} \cdot 3^{c_2} = 3^{c_2} = 1 \Rightarrow c_2 = 0$ \\
	$c_1 = c_2 = 0$ contradicts our assumption $\Rightarrow \{f_1, f_2\}$ are linearly independent
	\item $\{f_1,f_3\}$ \\
	$\{f_1,f_3\}$ are linearly dependent, if we can find $c_1 ,c_2 \in \R$ that are not both $0$ such that \\
	$\left( c_1 \odot f_1(x) \right) \oplus \left( c_2 \odot f_3(x) \right) = 1$ \\
	$x = a$: $\left( c_1 \odot 2 \right) \oplus \left( c_2 \odot 4 \right) = 1 \Rightarrow 2^{c_1} \cdot 4^{c_2} = 2^{c_1 + 2 c_2} = 1$ \\
	$x = b$: $\left( c_1 \odot 1 \right) \oplus \left( c_2 \odot 1 \right) = 1 \Rightarrow 1^{c_1} \cdot 1^{c_2} = 1^{c_1 + c_2} = 1$ \\
	$1^{c_1 + c_2} = 1$ implies $c_1, c_2$ can be any real number. We can choose $c_1 = 2, c_2 = -1$ to satisfy the two conditions above. Therefore, $\{f_1,f_3\}$ are linearly dependent.
\end{itemize}

\noindent {\bf Solution 1.3} \\

$\forall a_1, a_2 \in \R, \forall f_1, f_2 \in F$, 
\begin{align*}
\varphi\left((a_1 \odot f_1) \oplus (a_2 \odot f_2) \right) 
&= \varphi({f_1}^{a_1}{f_2}^{a_2}) = \sqrt{{f_1}^{a_1}{f_2}^{a_2}} = \sqrt{{f_1}^{a_1}}\sqrt{{f_2}^{a_2}} \\ 
&= {\sqrt{f_1}}^{a_1}{\sqrt{f_2}}^{a_2} = \left(a_1 \odot \sqrt{f_1}\right)\left(a_2 \odot \sqrt{f_2}\right) \\
&= \left(a_1 \odot \varphi(f_1)\right)\left(a_2 \odot \varphi{f_2} \right)= \left(a_1 \odot \varphi(f_1)\right) \oplus \left(a_2 \odot \varphi(f_2)\right)
\end{align*}
Therefore, $\varphi$ is a linear map over the space $F$ on $(F, \mathbb{R}, \oplus, \odot)$.

\clearpage

\noindent{\bf Exercise 2. (Range and null space {\bf$[40\, \text{points}]$})}\\
Let $(F, +, \cdot)$ be a field and consider the linear maps $\mathcal{A} : (F^n, F) \rightarrow (F^m, F)$ and  $\mathcal{B} : (F^m, F) \rightarrow (F^p, F)$. Show, without using the matrix representation of linear maps, that:

\begin{enumerate}
	\item {\bf$[10\, \text{points}]$} $0 \leq \dim(\range(\mathcal{A})) \leq \min\{m, n\}$.
	\item {\bf$[15\, \text{points}]$} $\dim(\range(\mathcal{A})) + \dim(\range(\mathcal{B})) - m \leq \dim(\range(\mathcal{B}\circ\mathcal{A}))$
	\item {\bf$[15\, \text{points}]$} $\dim(\range(\mathcal{B}\circ\mathcal{A})) \leq \min\{\dim(\range(\mathcal{A})),\dim(\range(\mathcal{B}))\}$. 
\end{enumerate}


\noindent {\bf Solution 2.1} \\
The left inequality is trivial since the number of vectors of a basis should definitely be non-negative, i.e., $0 \leq \dim(\range(\mathcal{A}))$. For the right inequality, we first recall a lemma.
\begin{lemma}
	If $(V,F)$ has dimension $n$ then any set of $n+1$ or more vectors is linearly dependent.
\end{lemma}
Next, we prove $\dim(\range(\A)) \leq m$. Assume, for the sake of contradiction, $\dim(\range(\A)) = p > m$, \\
and a basis of $\range(\A)$ is $\{v_1,\dots,v_p\}$, which are linearly independent by definition. \\
The fact that linearly independent vectors $\{v_1,\dots,v_p\} \subseteq \range(\A) \subseteq (V,F)$ and the dimension of $(V,F)$ is $m<p$ contradict the lemma. Therefore $\range(\A) \leq m$. \\

Finally, we prove $\dim(\range(\A)) \leq n$. $\forall v \in \range(\A), \exists u \in U, \A(u)=v$. Let $\{e_1,\dots,e_n\}$ be a basis of $U$. $\forall u \in U, \exists a_1, \dots, a_n, u = a_1 e_1 + \cdots + a_n e_n$, where $a_i$ are not all $0$. \\
Therefore, $\forall v \in \range(\A), v = \A (u) = \A \left(a_1 e_1 + \cdots + a_n e_n\right) = a_1 \A(e_1) + \cdots + a_n \A(e_n)$. If $\{\A(e_1), \dots, \A(e_n)\}$ are linearly independent, then $\dim(\range(\A)) = n$, otherwise $\dim(\range(\A)) < n$. \\

$\begin{cases}
\dim(\range(\A)) \leq n \\
\dim(\range(\A)) \leq m
\end{cases} \Rightarrow$ $\dim(\range(\A)) \leq \min \{m,n\}$, which completes the proof. \\

\vspace{0.6em}

\noindent {\bf Solution 2.2} \\
Define $\tilde{\B}$ as a restriction of $\B$, $\tilde{\B}:\range(\A) \rightarrow F^p$ \\
$\range(A)\subseteq F^m \Rightarrow \nullspace(\tilde{\B}) \subseteq \nullspace(\B)$ \\
Further, $\range(\tilde{\B}) = \range(\B\circ\A)$ by construction
\begin{flalign*}
\dim(\range(\A)) &= \dim(\nullspace(\tilde{\B})) + \dim(\range(\tilde{\B})) &\\
&\leq \dim(\nullspace(\B)) + \dim(\range(\tilde{\B})) &\\
&= \dim(\nullspace(\B)) + \dim(\range(\B\circ\A)) &\\
&= m-\dim(\range(\B)) + \dim(\range(\B\circ\A))
\end{flalign*}
Therefore, $\dim(\range(\mathcal{A})) + \dim(\range(\mathcal{B})) - m \leq \dim(\range(\mathcal{B}\circ\mathcal{A}))$. \\

\noindent {\bf Solution 2.3} \\
$\range(\B \circ \A) \subseteq \range(\B) \Rightarrow \dim(\range(\B \circ \A)) \leq \dim(\range(\B))$ \\
Let $\{v_1, \dots, v_r\}$ be a set of basis of $\range(\A)$. \\
$\range(\B \circ \A) = \{l \in F^p | \exists w \in \range(\A), \B (w) = l\}$. \\
$\range(\A) = span (\{v_1, \dots, v_r\}) \Rightarrow \forall w \in \range(\A), \exists c_1, \dots, c_r \in F, w = c_1 v_1 + \cdots + c_r v_r$ \\
$\B (w) = c_1 \B (v_1) + \cdots + c_r \B (v_r)$. \\
$\forall l \in F^p, \exists w \in \range(\A), \exists c_1, \dots, c_r, l=\B(w)=\B(c_1 v_1 + \cdots + c_r v_r) = c_1 \B(v_1) + \cdots + c_r \B(v_r)$ \\
If $\{\B(v_1),\dots,\B(v_r)\}$ are linearly independent, $\dim(\range(\B \circ \A)) = r = \dim(\range(\A))$, \\
otherwise $\dim(\range(\B \circ \A)) < r = \dim(\range(\A))$. \\

$\begin{cases}
\dim(\range(\B \circ \A)) \leq \dim(\range(\A)) \\
\dim(\range(\B \circ \A)) \leq \dim(\range(\B))
\end{cases} \Rightarrow \dim(\range(\B \circ \A)) \leq \min \{\dim(\range(\A)),\dim(\range(\B))\}$

\clearpage

\noindent{\bf Exercise 3. (Linear maps and matrix representations {\bf$[20\, \text{points}]$})}\\
Consider a linear map $\A:(U,F) \rightarrow (U,F)$
where $U$ has finite dimension $n$. 
\begin{enumerate}
	\item {\bf$[10\, \text{points}]$} Assume there exists a basis
	$\nu_i$, $i=1, \ldots, n$ for $U$ such that ${\mathcal A}(\nu_n)=\lambda
	\nu_n$ and ${\mathcal A}(\nu_i)=\lambda \nu_i+\nu_{i+1}$, $i=1, \ldots,
	n-1$. Derive the representation of ${\mathcal A}$ with respect to this~basis.
	\item {\bf$[10\, \text{points}]$} Assume there exists a vector $b
	\in U$ such that the set $\{b, {\mathcal A}(b), {\mathcal
		A}\circ {\mathcal A}(b), \ldots, {\mathcal A}^{n-1}(b)\}$ is linearly independent. Derive the representation of ${\mathcal A}$
	with respect to this basis.\\
\end{enumerate}

{\bf Solution 3.1}
\begin{align*}
y_1 = \A(\nu_1) &= \lambda \nu_1 + \nu_2 \Rightarrow a_{11} = \lambda, a_{21} = 1 \\
y_2 = \A(\nu_2) &= \lambda \nu_2 + \nu_3 \Rightarrow a_{22} = \lambda, a_{32} = 1\\
		  &\; \; \vdots \\
y_{n-1} = \A(\nu_{n-1}) &= \lambda \nu_{n-1} + \nu_n \Rightarrow a_{n-1,n-1} = \lambda, a_{n,n-1} = 1\\
y_n = \A(\nu_n) &= \lambda \nu_n \Rightarrow a_{nn} = \lambda \\
&\Rightarrow A = \left[\begin{matrix}
\lambda & & & & \\
1 & \lambda & & & \\
& 1 & \ddots & & \\
& & \ddots & \lambda & \\
& & & 1 & \lambda
\end{matrix}\right]
\end{align*}


{\bf Solution 3.2} \\

$\A:(U,F) \rightarrow (U,F) \Rightarrow \A^n(b) \in U$. \\
Since $\{b, {\A}(b), {\A} \circ {\A}(b), \ldots, {\A}^{n-1}(b)\}$ are linearly independent and $\dim(U) = n$, \\
$\exists c_1,\dots,c_n \in F$, which are not all zero, such that $\A^n(b) = c_1b + \cdots + c_n\A^{n-1}(b)$

\begin{align*}
y_1 = \A(b) &= \A(b) \Rightarrow a_{21} = 1 \\
y_2 = \A(\A(b)) &= \A^2(b) \Rightarrow a_{32} = 1\\
&\; \; \vdots \\
y_{n-1} = \A(\A^{n-2}(b)) &= \A^{n-1}(b) \Rightarrow a_{n,n-1} = 1 \\
y_n = \A(\A^{n-1}(b)) &= \A^n(b) \Rightarrow a_{jn} = c_j, j=1,\dots,n \\
&\Rightarrow A = \left[\begin{matrix}
 & & & & c_1\\
1 &  & & & \vdots \\
& 1 & & & \vdots\\
& & \ddots & & \vdots \\
& &  & 1 & c_n
\end{matrix}\right]
\end{align*}

\end{document}
