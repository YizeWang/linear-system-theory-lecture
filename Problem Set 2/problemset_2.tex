\documentclass[fleqn,12pt, a4paper]{article}

\usepackage{psfrag}
\usepackage[dvips]{epsfig}
\usepackage{amsmath,amssymb}
\usepackage{theorem}
\usepackage{epsfig}
\usepackage{color}

\def\R{{\mathbb R}}
\renewcommand{\Re}{{\mathbb R}}
\def\N{{\mathbb N}}
\def\C{{\mathbb C}}
\def\Z{{\mathbb Z}}
\def\Q{{\mathbb Q}}
\def\C{\mathbb{C}}


\def\I{\mathcal{I}}
\def\A{\mathcal{A}}
\def\B{\mathcal{B}}
\def\U{\mathcal{U}}
\def\V{\mathcal{V}}
\def\W{\mathcal{W}}
\def\F{\mathcal{F}}
\def\Nscr{{\mathcal{N}}}
\def\Mscr{{\mathcal{M}}}

%------------------------------------------------------------
% Theorem like environments
%
\newtheorem{theorem}{Theorem}
\newtheorem{corollary}{Corollary}
\newtheorem{lemma}{Lemma}
\newtheorem{proposition}{Proposition}
%\theoremstyle{plain}
\newtheorem{acknowledgement}{Acknowledgement}
\newtheorem{algorithm}{Algorithm}
\newtheorem{axiom}{Axiom}
\newtheorem{case}{Case}
\newtheorem{claim}{Claim}
\newtheorem{conclusion}{Conclusion}
\newtheorem{condition}{Condition}
\newtheorem{criterion}{Criterion}
\theoremstyle{definition}
\newtheorem{definition}{Definition}
\newtheorem{example}{Example}
\newtheorem{exercise}{Exercise}
\newtheorem{notation}{Notation}
\newtheorem{problem}{Problem}
\newtheorem{remark}{Remark}
\newtheorem{solution}{Solution}
\newtheorem{summary}{Summary}
\numberwithin{equation}{section}
%--------------------------------------------------------

\def\bmat{\left[ \begin{array}}
\def\emat{\end{array} \right]}

\newcommand{\spanop}{\mathop{\rm span}}
\def\bmat{\left[ \begin{array}}
\def\emat{\end{array} \right]}

\headheight 8mm
\headsep 16mm
\topmargin  -20mm
\oddsidemargin -.15in
\textwidth 160mm
\textheight 240mm
\baselineskip 7mm
\parindent 0mm
%\pagestyle{headings}
%\markright{EIST: Paper 3 Examples Paper 1 }

\def\domain{{\mbox{\em domain}}}
\def\dim{{\mbox{\em dim}}}
\def\range{{\mbox{\sc Range}}}
\def\null{{\mbox{\sc Null}}}
\def\nullity{{\mbox{\sc Nullity}}}
\def\rank{{\mbox{\sc Rank}}}

%\def\nulls{\mbox{\sc Null}}
%def\range{\mbox{\sc Range}}
%\newcommand{\spn}{\mbox{\sc Span}}
%\def\rank{\mbox{\sc Rank}



\def\ifpart{\textbf{(if) }}
\def\onlyifpart{\textbf{(only if) }}
\def\solution{\noindent{\it Solution.} }

\begin{document}
\thispagestyle{plain}

%%%%%%%%%%%%%%%%%%%%%%%%%%%%%%%%%%%%%%%%%%%%%%%%%%%%%%%%%%%%%%%%%%%%%%
\vspace*{-1.5cm}
{\noindent \rule{15.8cm}{.3mm} \\[.3cm]}
\begin{center} \bf
{\large Linear System Theory \medskip
\\
Problem Set 2 \\
Normed Spaces, ODEs, and Linear Time-Varying Systems\medskip
\\ Issue date: October 7, 2019
\\ Due date: October 21, 2019}
\end{center}
\rule{15.8cm}{.3mm} \\[0cm]
%%%%%%%%%%%%%%%%%%%%%%%%%%%%%%%%%%%%%%%%%%%%%%%%%%%%%%%%%%%%%%%%%%%%%%

%==========================================================================================================
% 2016,2
\noindent
{\bf Exercise 1. (Norms, {\bf$[45\, \text{points in total}]$})}
\noindent
\begin{enumerate}
\item {\bf$[15\, \text{points}]$}
Let $C([t_0, t_1], \R^n)$ be the space of all continuous functions from $[t_0, t_1]$ to $\R^n$. Prove that for $f \, \in \, C([t_0, t_1], \R^n)$,
 $\left\lVert f \right\rVert_\infty := \max\limits_{t \in [t_0, t_1]} \|f(t)\|_p \, $ satisfies the axioms of the norm, where $\|x\|_p$ is the $p$-norm of $x \in \R^n$.

\item {\bf$[10\, \text{points}]$}
Given a matrix $A \in \R^{m \times n}$, verify that the induced matrix norms $\|A\|_2 \, ,\|A\|_\infty$ are equivalent, by showing that they satisfy the following inequalities: 
\begin{displaymath}
\dfrac{1}{\sqrt{n}} \|A\|_\infty \leq \|A\|_2 \leq \sqrt{m} \|A\|_\infty.
\end{displaymath}

{\em Hint:} The induced $p$-norm of a matrix $A$ is given by:
\begin{displaymath}
\|A\|_p=\sup\limits_{\|x\|_p \neq 0} \dfrac{\|Ax\|_p}{\|x\|_p}
\end{displaymath}

\item {\bf$[20\, \text{points}]$}
Consider a set of functions $f_n$ in $C([0,1], \R)$ defined as:
\begin{displaymath}
f_n:[0,1] \rightarrow \R \quad\text{s.t.}\quad
f_n(x) = 
\begin{cases}
n -n^2x & 0\leq x\leq \frac{1}{n}\\
0 & \text{elsewhere}
\end{cases}
\end{displaymath}
for $n\in\N$. Compute the 1-norm, 2-norm and $\infty$-norm of the functions for each n, as defined below:
\begin{displaymath}
\left\lVert f \right\rVert_1 := \int_0^1 \left\lvert f(x) \right\rvert dx\, , \quad
\left\lVert f \right\rVert_2 := \sqrt{\int_0^1 \left\lvert f(x) \right\rvert^2 dx}\, , \quad
\left\lVert f \right\rVert_\infty := \max\limits_{t \in [0, 1]} |f(t)|.
\end{displaymath}
Based on your computations, what can you say about equivalence of these norms?  


\end{enumerate} 

%%%%%%%%%%%%%%%%%%%%%%%%%%%%%%%%%%%%%%%%%%%%%%%%%%%%%%%%%%%%%%%%%%%%%%
\bigskip
\clearpage
%2017, 2
{\bf Exercise 2. (Banach fixed point theorem  {\bf$[25\, \text{points in total}]$})}
\begin{enumerate}
\item {\bf$[20\, \text{points}]$} Let $\left(X, \left\lVert\cdot\right\rVert\right)$ be a Banach space, and $f:X\rightarrow X$. Assume that there exists $\alpha\in[0,\, 1)$ such that, for all $x,y\in X$,
\begin{displaymath}
\left\lVert f(x) - f(y) \right\rVert
\leq
\alpha
\left\lVert x - y \right\rVert.
\end{displaymath}
Show that there exists a unique point $\bar{x}$ such that $f(\bar{x}) = \bar{x}$.

\noindent
{\em Hint:} 
Given an arbitrary initial point $x$,
consider the sequence of iterates 
$f^{[n]}(x) = f(f^{[n-1]}(x))$, where the first iterate is given by $f^{[0]}(x)=x$. You can start by showing that this sequence is Cauchy.

\item {\bf$[5\, \text{points}]$} Now assume $f$ is a linear map. Given the condition in the first part of Exercise 2, what can you conclude about the induced norm of $f$?

\end{enumerate}

%%%%%%%%%%%%%%%%%%%%%%%%%%%%%%%%%%%%%%%%%%%%%%%%%%%%%%%%%%%%%%%%%%%%%%
\bigskip

\noindent
% 2017, 2
{\bf Exercise 3. (Ordinary differential equations {\bf$[30\, \text{points in total}]$})}
\begin{enumerate}
	\item {\bf$[12\, \text{points}]$} Consider the following ordinary differential equation (ODE)
	\begin{displaymath}
	\begin{bmatrix}\dot{x}_1(t)\\ \dot{x}_2(t)\end{bmatrix}= \begin{bmatrix}
	-x_1(t)+e^t\, \cos(x_1(t)-x_2(t)) \\ -x_2(t)+\sin(x_1(t)-x_2(t))
	\end{bmatrix},
	\end{displaymath}
	where $x_i(t) \in \R,\, \forall i$. Prove or disprove the following statements:
	\begin{enumerate}
		\item This system is globally Lipschitz,
		\item This system admits a unique solution.
		\\
		{\em Hint:} You may assume that functions with bounded derivatives are Lipschitz.
	\end{enumerate}
\item {\bf$[18\, \text{points}]$} Consider the following ordinary differential equation (ODE)
\begin{displaymath}
\begin{bmatrix}\dot{x}_1(t)\\ \dot{x}_2(t)\end{bmatrix}= \begin{bmatrix}
-3 \sin(t)\, x_1(t) + x_1(t)\, x_2(t) \\ -2\, x_2(t)
\end{bmatrix},
\end{displaymath}
where $x_i(t) \in \R,\, \forall i$. Prove or disprove the following statements:
\begin{enumerate}
	\item This system is globally Lipschitz,
	\item This system admits a unique solution.
\end{enumerate}
\end{enumerate}

%%%%%%%%%%%%%%%%%%%%%%%%%%%%%%%%%%%%%%%%%%%%%%%%%%%%%%%%%%%%%%%%%%%%%%


%%%%%%%%%%%%%%%%%%%%%%%%%%%%%%%%%%%%%%%%%%%%%%%%%%%%%%%%%%%%%%%%%%%%%%

\end{document}
