% style settings
\documentclass[a4paper,10.5pt]{article}
\headsep 16mm
\parindent 0mm
\headheight 8mm
\topmargin -20mm
\textwidth 160mm
\textheight 240mm
\baselineskip 7mm
\oddsidemargin -.15in

% import packages
\usepackage{color}
\usepackage{epsfig}
\usepackage{psfrag}
\usepackage{amsmath}
\usepackage{amssymb}
\usepackage{theorem}
\usepackage{multicol}
\usepackage{mathtools}

% define shortcuts
\def\dim{{\mbox{\em dim}}}
\def\ifpart{\textbf{(if) }}
\def\rank{{\mbox{\sc Rank}}}
\def\null{{\mbox{\sc Null}}}
\def\range{{\mbox{\sc Range}}}
\def\domain{{\mbox{\em domain}}}
\def\nullity{{\mbox{\sc Nullity}}}
\def\onlyifpart{\textbf{(only if) }}
\def\solution{\noindent{\it Solution.}}

% define new commands and environment
\newtheorem{lemma}{Lemma}
\newtheorem{proof}{Proof}
\newcommand{\de}{\mathrm{d}}
\newcommand{\R}{\mathbb{R}}
\newcommand{\N}{\mathbb{N}}
\newcommand{\I}{\mathbb{I}}
\newcommand{\A}{\mathcal{A}}
\newcommand{\B}{\mathcal{B}}
\newcommand{\defeq}{\vcentcolon=}
\newcommand{\dt}{\, \mathrm{d} t}
\newcommand{\dx}{\, \mathrm{d} x}
\newcommand{\supx}{\sup_{x\neq0}}
\newcommand{\abs}[1]{\left|#1\right|}
\newcommand{\spanop}{\mathop{\rm span}}
\newcommand{\norm}[1]{\left\lVert#1\right\rVert}
\newenvironment{mymat}{\left[\begin{matrix}}{\end{matrix}\right]}

\begin{document}
\thispagestyle{plain}

% title
\vspace*{-1.5cm}
\noindent \rule{15.8cm}{.3mm} \\[.3cm]
\begin{center} \bf
{\large Linear System Theory \medskip \\
Problem Set 2 \\
Normed Spaces, ODEs, and Linear Time-Varying Systems \medskip \\
Issue date: October 7, 2019 \\
Due date: October 21, 2019}
\end{center}
\rule{15.8cm}{.3mm} \\[0cm]

% exercise 1
\noindent {\bf Exercise 1. (Norms, {\bf$[45\, \text{points in total}]$})}
\begin{enumerate}
\item {\bf$[15\, \text{points}]$}
Let $C([t_0, t_1], \R^n)$ be the space of all continuous functions from $[t_0, t_1]$ to $\R^n$. Prove that for $f \, \in \, C([t_0, t_1], \R^n)$,
 $\left\lVert f \right\rVert_\infty := \max\limits_{t \in [t_0, t_1]} \|f(t)\|_p \, $ satisfies the axioms of the norm, where $\|x\|_p$ is the $p$-norm of $x \in \R^n$.
 
\item {\bf$[10\, \text{points}]$}
Given a matrix $A \in \R^{m \times n}$, verify that the induced matrix norms $\|A\|_2 \, ,\|A\|_\infty$ are equivalent, by showing that they satisfy the following inequalities:
\begin{displaymath}
\dfrac{1}{\sqrt{n}} \|A\|_\infty \leq \|A\|_2 \leq \sqrt{m} \|A\|_\infty.
\end{displaymath}

{\em Hint:} The induced $p$-norm of a matrix $A$ is given by:
\begin{displaymath}
\|A\|_p=\sup\limits_{\|x\|_p \neq 0} \dfrac{\|Ax\|_p}{\|x\|_p}
\end{displaymath}

\item {\bf$[20\, \text{points}]$}
Consider a set of functions $f_n$ in $C([0,1], \R)$ defined as:
\begin{displaymath}
f_n:[0,1] \rightarrow \R \quad\text{s.t.}\quad
f_n(x) = 
\begin{cases}
n -n^2x & 0\leq x\leq \frac{1}{n}\\
0 & \text{elsewhere}
\end{cases}
\end{displaymath}

for $n\in\N$. Compute the 1-norm, 2-norm and $\infty$-norm of the functions for each n, as defined below:
\begin{displaymath}
\left\lVert f \right\rVert_1 := \int_0^1 \left\lvert f(x) \right\rvert dx\, , \quad
\left\lVert f \right\rVert_2 := \sqrt{\int_0^1 \left\lvert f(x) \right\rvert^2 dx}\, , \quad
\left\lVert f \right\rVert_\infty := \max\limits_{t \in [0, 1]} |f(t)|.
\end{displaymath}

Based on your computations, what can you say about equivalence of these norms?  
\end{enumerate} 

\clearpage

\noindent \textbf {Solution 1.1} \\
\begin{itemize}
	\item $\forall f, g \in C([t_0, t_1], \R^n), \|f+g\|_\infty \leq \|f\|_\infty + \|g\|_\infty$
	\begin{align*}
		\forall f, g \in C([t_0, t_1], \R^n), \|f+g\|_\infty &= \max_{t \in [t_0, t_1]} \| f + g\|_p \\
		&\leq  \max_{t \in [t_0, t_1]} \{\| f \|_p + \|g\|_p\} \\
		&\leq \max_{t \in [t_0, t_1]} \| f \|_p + \max_{t \in [t_0, t_1]} \|g\|_p \\
		&= \|f\|_\infty + \|g\|_\infty
	\end{align*}
	\item $\forall f \in C([t_0, t_1], \R^n), \forall a \in \R, \|af\| = \abs{a}\|f\|$
	\begin{align*}
		\forall f \in C([t_0, t_1], \R^n), \forall a \in \R, \|af\| &= \max_{t \in [t_0, t_1]} \| af \|_p \\
		&= \max_{t \in [t_0, t_1]} \abs{a} \| f \|_p \\
		&= \abs{a} \max_{t \in [t_0, t_1]} \| f \|_p \\
		&= \abs{a} \| f \|_\infty \\
	\end{align*}
	\item $\|f\|=0 \Leftrightarrow f = 0$ \\
	$\Rightarrow: \|f\|=0\Rightarrow\max_{t \in [t_0, t_1]} \| f \|_p = 0$, $\|f\|_p \geq 0 \Rightarrow \|f\|_p = 0, \forall t \in [t_0, t_1] \Rightarrow f = 0$ \\
	$\Leftarrow: f = 0 \Rightarrow \|f\|_p = 0, \forall t \in [t_0, t_1] \Rightarrow \max_{t \in [t_0, t_1]} \| f \|_p = 0 \Rightarrow \|f\|=0$ \\
\end{itemize} 

\noindent \textbf {Solution 1.2} \\

Define $x_{max}$ and $\left(Ax\right)_{max}$ to simplify further notation:
\begin{align*}
	x_{max} &= \max_{i\in\{1,\dots,n\}} \abs{x_i} \\
	\left(Ax\right)_{max} &= \max_{j\in\{1,\dots,m\}}\abs{\left(Ax\right)_j}
\end{align*}
Then, start with $\|A\|_2^2$:
\begin{align*}
\|A\|_2^2&=\left(\supx\frac{\|Ax\|_2}{\|x\|_2}\right)^2=\supx\frac{\|Ax\|_2^2}{\|x\|_2^2} = \supx \frac{\sum_{i=1}^{m} \left(Ax\right)_i^2}{\sum_{i=1}^{n} x_i^2} \geq \supx \frac{\sum_{i=1}^{m} \left(Ax\right)_i^2}{\sum_{i=1}^{n} x_{max}^2} \\
&=\supx \frac{1}{n}\frac{\sum_{i=1}^{m} \left(Ax\right)_i^2}{x_{max}^2} \geq \supx \frac{1}{n}\frac{\left(Ax\right)_{max}^2}{x_{max}^2} = \supx \frac{1}{n} \left(\frac{\left(Ax\right)_{max}}{x_{max}}\right)^2 \\
&= \frac{1}{n}  \left(\supx\frac{\left(Ax\right)_{max}}{x_{max}}\right)^2 =\frac{1}{n} \left(\supx \frac{\|Ax\|_\infty}{\|x\|_\infty}\right)^2= \frac{1}{n}\|A\|_\infty^2 \\
&\Rightarrow \|A\|_2 \geq \frac{1}{\sqrt{n}}\|A\|_\infty \\
\|A\|_2^2&=\left(\supx\frac{\|Ax\|_2}{\|x\|_2}\right)^2=\supx\frac{\|Ax\|_2^2}{\|x\|_2^2} = \supx \frac{\sum_{i=1}^{m} \left(Ax\right)_{i}^2}{\sum_{i=1}^{n} x_i^2} \leq \supx \frac{\sum_{i=1}^{m} \left(Ax\right)_{max}^2}{\sum_{i=1}^{n} x_{i}^2} \\&= \supx \frac{m \left(Ax\right)_{max}^2}{\sum_{i=1}^{n} x_{i}^2} = m\supx \frac{\left(Ax\right)_{max}^2}{\sum_{i=1}^{n} x_{i}^2} \leq m\supx \frac{\left(Ax\right)_{max}^2}{x_{max}^2} = m\supx \left(\frac{\left(Ax\right)_{max}}{x_{max}}\right)^2 \\
&=m\left(\supx \frac{\left(Ax\right)_{max}}{x_{max}}\right)^2 = m\left(\supx \frac{\|Ax\|_\infty}{\|x\|_\infty}\right)^2 = m\|A\|_\infty \\
&\Rightarrow \|A\|_2 \leq \sqrt{m}\|A\|_\infty
\end{align*}

\clearpage

\noindent \textbf {Solution 1.3} \\
\begin{align*}
\forall x \in \left[0,\frac{1}{n}\right], \forall n \in \N, n - n^2 x \geq 0 \Rightarrow f_n (x) \geq 0, \forall x \in [0,1] \Rightarrow \lvert f_n(x) \rvert = f_n(x), \forall x \in [0,1] \\
\end{align*}

\textbf{Compute norms:}
\begin{itemize}
\item $\|f\|_1$
\begin{align*}
	\|f\|_1 = \int_0^1 \lvert f_n(x)\rvert \dx=\int_0^1f_n(x)\dx = \int_0^{\frac{1}{n}}\left(n-n^2x\right) \dx = nx-\frac{1}{2}n^2x^2 \bigg\rvert_0^{\frac{1}{n}} = \frac{1}{2}
\end{align*}
\item $\|f\|_2$
\begin{align*}
\|f\|_2^2&=\int_0^1 \lvert f_n(x)\rvert ^2 \dx = \int_0^{\frac{1}{n}}\left(n-n^2x\right)^2\dx \\
&=\int_0^{\frac{1}{n}}\left(n^4x^2-2n^3x+n^2\right) \dx = \frac{1}{3}n^4x^3-n^3 x^2 + n^2 x \bigg\rvert_0^{\frac{1}{n}}=\frac{1}{3}n \\
\|f\|_2 &= \sqrt{\frac{1}{3}n}
\end{align*}
\item $\|f\|_\infty$
\begin{align*}
\|f\|_\infty = \max_{t \in [0,1]} \lvert f_n(t) \rvert = \max_{t\in \left[0,1/n\right]} n-n^2t=n
\end{align*}

\end{itemize}

\textbf{Equivalence test:}
\begin{itemize}
\item Equivalence of $\|f_n\|_1$ and $\|f_n\|_\infty$
\begin{align*}
\forall m_l > 0, \exists f_n \in C([0,1], \R) \text{ and } n > \frac{1}{2m_l}, m_l\norm{f_n}_\infty = m_l \cdot n > m_l \cdot \frac{1}{2m_l} = \frac{1}{2} = \norm{f_n}_1
\end{align*}
Therefore, $\|f_n\|_1$ and $\|f_n\|_\infty$ are not equivalent.
\item Equivalence of $\|f_n\|_2$ and $\|f_n\|_\infty$
\begin{align*}
\forall m_l > 0, \exists f_n \in C([0,1], \R) \text{ and } n > \frac{1}{3m_l^2}, m_l\norm{f_n}_\infty = m_l \cdot n > \sqrt{\frac{1}{3n}} \cdot n = \sqrt{\frac{1}{3}n} = \norm{f_n}_2
\end{align*}
Therefore, $\|f_n\|_1$ and $\|f_n\|_\infty$ are not equivalent.

\item Equivalence of $\|f_n\|_1$ and $\|f_n\|_2$
\begin{align*}
\forall m_l > 0, \exists f_n \in C([0,1], \R) \text{ and } n > \frac{3}{4m_l^2}, m_l\norm{f_n}_2 = m_l \cdot \sqrt{\frac{1}{3}n} > m_l \cdot \sqrt{\frac{1}{3} \cdot \frac{3}{4m_l^2}} = \frac{1}{2} = \norm{f_n}_1
\end{align*}
Therefore, $\|f_n\|_1$ and $\|f_n\|_2$ are not equivalent.
\end{itemize}
To conclude, none of them is equivalent to another.

\clearpage

% exercise 2
{\bf Exercise 2. (Banach fixed point theorem  {\bf$[25\, \text{points in total}]$})}
\begin{enumerate}
\item {\bf$[20\, \text{points}]$} Let $\left(X, \left\lVert\cdot\right\rVert\right)$ be a Banach space, and $f:X\rightarrow X$. Assume that there exists $\alpha\in[0,\, 1)$ such that, for all $x,y\in X$,
\begin{displaymath}
\left\lVert f(x) - f(y) \right\rVert
\leq
\alpha
\left\lVert x - y \right\rVert.
\end{displaymath}
Show that there exists a unique point $\bar{x}$ such that $f(\bar{x}) = \bar{x}$.

\noindent {\em Hint:} 
Given an arbitrary initial point $x$,
consider the sequence of iterates 
$f^{[n]}(x) = f(f^{[n-1]}(x))$, where the first iterate is given by $f^{[0]}(x)=x$. You can start by showing that this sequence is Cauchy.

\item {\bf$[5\, \text{points}]$} Now assume $f$ is a linear map. Given the condition in the first part of Exercise 2, what can you conclude about the induced norm of $f$?

\end{enumerate}

\noindent \textbf {Solution 2.1}
\begin{align*}
	f^{[0]}(x) &= x \\
	f^{[1]}(x) &= f\left(f^{[0]}(x)\right) = f(x) \\
	f^{[2]}(x) &= f\left(f^{[1]}(x)\right) = f(f(x)) \\
			   & \;\; \vdots \\
	f^{[n]}(x) &= f\left(f^{[n-1]}(x)\right) = f(\cdots f(f(x)))		  
\end{align*}
$\forall p\geq q\geq 0, p,q\in N:$
\begin{align*}
	\|f^{[p]}(x)-f^{[q]}(x)\| &= \norm{f\left(f^{[p-1]}(x)\right)-f\left(f^{[q-1]}(x)\right)} \\
	&\leq \alpha \norm{f^{[p-1]}(x)-f^{[q-1]}(x)} \\
	& \; \; \vdots \\
	&\leq \alpha^{q} \norm{f^{[p-q]}(x)-f^{[0]}(x)} \\
	&= \alpha^q \norm{f^{[p-q]}(x)-x}
\end{align*}
$\forall k \in N:$
\begin{align*}
	\norm{f^{[k]}(x)-x} &= \norm{f^{[k]}(x)-f(x)+f(x)-x} \\
	&\leq \norm{f^{[k]}(x)-f(x)}+\norm{f(x)-x} \\
	&\leq \alpha\norm{f^{[k-1]}(x)-x}+\norm{f(x)-x} \\
	&\leq \alpha^2\norm{f^{[k-2]}(x)-x}+\left(1+\alpha\right)\norm{f(x)-x} \\
	& \;\; \vdots \\
	&\leq \alpha^{k-1}\norm{f^{[1]}(x)-x} + \left(1+\alpha+\cdots+\alpha^{k-2}\right)\norm{f(x)-x} \\
	&\leq \left(1+\alpha+\cdots+\alpha^{k-2}+\alpha^{k-1}\right)\norm{f(x)-x} \\
	&=\frac{1-\alpha^k}{1-\alpha}\norm{f(x)-x}
\end{align*}

\clearpage

$\forall \epsilon>0,\exists N = \bigg\lceil \log_\alpha\left(\frac{\epsilon(1-\alpha)}{\norm{f(x)-x}}+\alpha^m\right) \bigg\rceil \in \N \ (\lceil t \rceil \text{ means the smallest integer greater than } t),\forall m \geq N,$
\begin{align*}
\norm{f^{[m]}(x)-f^{[N]}(x)} 
&\leq \alpha^N \norm{f^{[m-N]}(x)-x} \\
&\leq \alpha^N \cdot \frac{1-\alpha^{m-N}}{1-\alpha} \cdot \norm{f(x)-x} \\
&= \frac{\alpha^N-\alpha^{m}}{1-\alpha} \norm{f(x)-x} \\
N&>\log_\alpha\left(\frac{\epsilon(1-\alpha)}{\norm{f(x)-x}}+\alpha^m\right) \\
\Rightarrow \norm{f^{[m]}(x)-f^{[N]}(x)} &< \epsilon
\end{align*}
Therefore, $\{f^{[i]}\}_{i=0}^{\infty}$ is a Cauchy sequence.
Note that given an arbitrary initial point $x$, we can compute the value of $\|f(x)-x\|$ so we treat $\|f(x)-x\|$ as a known constant in the proof above. Because $\left(X, \left\lVert\cdot\right\rVert\right)$ is a Banach space, $\{f^{[i]}\}_{i=0}^{\infty}$ converges to a point $\bar{x}$. \\

Next, we prove the uniqueness: For the sake of contradiction, assume there exist $\bar{x}_1,\bar{x}_2 \in X, \bar{x}_1 \neq \bar{x}_2$ such that $f(\bar{x}_1)=\bar{x}_1$ and $f(\bar{x}_2)=\bar{x}_2$.
\begin{align*}
	\norm{f(\bar{x}_2)-f(\bar{x}_1)} =\norm{\bar{x}_2-\bar{x}_1} \leq \alpha\norm{\bar{x}_2-\bar{x}_1}	
\end{align*}
Since $\alpha \in [0,1)$, $\alpha \neq 1$. The above inequality never holds, which leads to contradiction, so $\bar{x}$ is unique. \\

\noindent \textbf {Solution 2.2} \\
Let $F$ be the representation of linear map $f(\cdot)$. $\forall x,y \in X$:
\begin{align*}
\norm{f(x)-f(y)} = \norm{Fx-Fy} = \norm{F(x-y)} \leq \alpha \norm{x-y} \Rightarrow \frac{\norm{F(x-y)}}{\norm{x-y}} &\leq \alpha
\end{align*}
$\forall u \in X, \exists x,y \in X$ such that $u = x-y$.
\begin{align*}
\norm{F} = \sup_{u\neq 0}\frac{\norm{Fu}}{\norm{u}} \leq \sup_{u\neq 0} \alpha = \alpha
\end{align*}
Therefore, the induced norm of $f$ is bounded by $\alpha$.

\clearpage

% exercise 3
{\bf Exercise 3. (Ordinary differential equations {\bf$[30\, \text{points in total}]$})}
\begin{enumerate}
	\item {\bf$[12\, \text{points}]$} Consider the following ordinary differential equation (ODE)
	\begin{displaymath}
	\begin{bmatrix}\dot{x}_1(t)\\ \dot{x}_2(t)\end{bmatrix}= \begin{bmatrix}
	-x_1(t)+e^t\, \cos(x_1(t)-x_2(t)) \\ -x_2(t)+\sin(x_1(t)-x_2(t))
	\end{bmatrix},
	\end{displaymath}
	where $x_i(t) \in \R,\, \forall i$. Prove or disprove the following statements:
	\begin{enumerate}
		\item This system is globally Lipschitz,
		\item This system admits a unique solution.
		\\
		{\em Hint:} You may assume that functions with bounded derivatives are Lipschitz.
	\end{enumerate}
\item {\bf$[18\, \text{points}]$} Consider the following ordinary differential equation (ODE)
\begin{displaymath}
\begin{bmatrix}\dot{x}_1(t)\\ \dot{x}_2(t)\end{bmatrix}= \begin{bmatrix}
-3 \sin(t)\, x_1(t) + x_1(t)\, x_2(t) \\ -2\, x_2(t)
\end{bmatrix},
\end{displaymath}
where $x_i(t) \in \R,\, \forall i$. Prove or disprove the following statements:
\begin{enumerate}
	\item This system is globally Lipschitz,
	\item This system admits a unique solution.
\end{enumerate}
\end{enumerate}

\vspace{0.5em}

\noindent \textbf {Solution 3.1}
\begin{align*}
	\dot{x} &= p(x,t) \\
	\frac{\partial p}{\partial x} &=  \begin{mymat}\frac{\partial p_1}{\partial x_1} & \frac{\partial p_1}{\partial x_2} \\
	\frac{\partial p_2}{\partial x_1} & \frac{\partial p_2}{\partial x_2}
	\end{mymat} = \begin{mymat}
	-1 - e^t \sin(x_1-x_2) & e^t \sin (x_1-x_2) \\
	\cos(x_1-x_2) & -1-\cos(x_1-x_2)
	\end{mymat} \\
	\abs{\frac{\partial p_1}{\partial x_1}} &=  \abs{-1 - e^t \sin(x_1-x_2)} \leq 1+\abs{e^t \sin(x_1-x_2)} \leq 1+e^t\\
	\abs{\frac{\partial p_1}{\partial x_2}} &= \abs{e^t \sin (x_1-x_2)} \leq e^t\\
	\abs{\frac{\partial p_2}{\partial x_1}} &= \abs{\cos(x_1-x_2)} \leq 1\\
	\abs{\frac{\partial p_2}{\partial x_2}} &=\abs{-1-\cos(x_1-x_2)}\leq 1 + \abs{-\cos(x_1-x_2)} \leq 2
\end{align*}
The function $p(x,t)$ is differentiable in $x$ and the derivative is bounded (since each element is bounded), which leads to the conclusion that $p(x,t)$ is globally Lipschitz. It is also obvious that $p(x,t)$ is continuous in $t$. From (a) we know that $p(x,t)$ is globally Lipschitz with respect to $x$. Per \textbf{Theorem~{3.6}}, the system admits a unique solution. \\

\clearpage

\noindent \textbf {Solution 3.2} \\
Consider two states
\begin{align*}
\hat{x} = \begin{mymat}c\\c\end{mymat}, \theta = \begin{mymat}0\\0\end{mymat}, c\gg 0
\end{align*}
\begin{align*}
\|p(\hat{x},t)-p(\theta,t)\|_\infty = \norm{\begin{mymat}
	-3\sin(t)c+c^2\\-2c
	\end{mymat}}_\infty&=-3\sin(t)c+c^2 \geq c^2-3c
\end{align*}
\begin{align*}
\forall k \in \R, \exists c=k+3,\|p(\hat{x},t)-p(\theta,t)\|_\infty \geq c^2-3c = kc = k\norm{\bar{x}-\theta}_\infty
\end{align*}
Therefore, the system is not globally Lipschitz continuous in $x$. \\

$p_2(x_2(t),t)=\dot{x}_2(t)=-2x_2(t) \Rightarrow x_2(t) = e^{-2t}x_2(0)$\\
$p_2(x_2(t),t)$ is piecewise continuous in $t$ and Lipschitz continuous in $x_2$ $\Rightarrow$ $x_2(t)$ admits unique solution. \\
$p_1(x_1(t),t)=\dot{x}_1(t)=-3\sin(t)x_1(t)+x_1(t)e^{-2t}x_2(0)=\left(e^{-2t}x_2(0)-3\sin(t)\right)x_1(t)$ \\
$p_1(x_1(t),t)$ is piecewise continuous in $t$ and Lipschitz continuous in $x_1$ $\Rightarrow$ $x_1(t)$ admits unique solution. \\

Therefore, the system admits unique solution.

\end{document}