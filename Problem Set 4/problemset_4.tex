% style settings
\documentclass[a4paper,10.5pt]{article}
\headsep 16mm
\parindent 0mm
\headheight 8mm
\topmargin -20mm
\textwidth 160mm
\textheight 240mm
\baselineskip 7mm
\oddsidemargin -.15in

% import packages
\usepackage{color}
\usepackage{epsfig}
\usepackage{psfrag}
\usepackage{amsmath}
\usepackage{amssymb}
\usepackage{theorem}
\usepackage{multicol}
\usepackage{mathtools}

% define shortcuts
\def\dim{{\mbox{\em dim}}}
\def\ifpart{\textbf{(if) }}
\def\rank{{\mbox{\sc Rank}}}
\def\nullspace{{\mbox{\sc Null}}}
\def\range{{\mbox{\sc Range}}}
\def\domain{{\mbox{\em domain}}}
\def\nullity{{\mbox{\sc Nullity}}}
\def\onlyifpart{\textbf{(only if) }}
\def\solution{\noindent{\it Solution.}}

% define new commands and environment
\newtheorem{lemma}{Lemma}
\newtheorem{proof}{Proof}
\newcommand{\de}{\mathrm{d}}
\newcommand{\R}{\mathbb{R}}
\newcommand{\N}{\mathbb{N}}
\newcommand{\I}{\mathbb{I}}
\newcommand{\A}{\mathcal{A}}
\newcommand{\B}{\mathcal{B}}
\newcommand{\defeq}{\vcentcolon=}
\newcommand{\dt}{\, \mathrm{d} t}
\newcommand{\dx}{\, \mathrm{d} x}
\newcommand{\supx}{\sup_{x\neq0}}
\newcommand{\abs}[1]{\left|#1\right|}
\newcommand{\spanop}{\mathop{\rm span}}
\newcommand{\norm}[1]{\left\lVert#1\right\rVert}
\newenvironment{mymat}{\left[\begin{matrix}}{\end{matrix}\right]}

\begin{document}
\thispagestyle{plain}

% title
\vspace*{-1.5cm}
{\noindent \rule{15.8cm}{.3mm} \\[.3cm]}
\begin{center} \bf
{\large Linear System Theory \medskip \\
Problem Set 4 \\
Stability of LTI and LTV, Inner product spaces \medskip
\\ Issue date: Nov. 12, 2019 \\ Due date: Nov. 26, 2019}
\end{center}
\rule{15.8cm}{.3mm} \\[1cm]

% exercise 1
\noindent {\bf Exercise 1, (Lyapunov stability and salmon extinction, $[$50 points in total$]$)}

Consider a system $\dot{x}(t)=f(x(t))$, where $f: \R^n \rightarrow \R^n$ is Lipschitz continuous. Suppose we have
\begin{align}
\frac{d}{dt}\big(x(t)^TPx(t)\big)\leq -x(t)^TQx(t)\,, \tag{1}
\end{align}
where $P$ and $Q$ are symmetric positive definite matrices (here the time-derivative is taken along the state trajectories of the system). 

\begin{itemize}
\item[1.] $[$20 points$]$ Prove that under  condition (1) the system is exponentially stable, in the sense that its solutions satisfy $\|x(t)\|\leq k$e$^{-\mu t} \|x(0)\|$ for some $k, \mu > 0$. Then, show that if $f(x) = Ax$, the condition (1) is equivalent to 
\begin{equation}
A^TP + PA \leq -Q \tag{2}\,.
\end{equation}

{\em Hints:} You can use the fact that, for any positive definite matrix $N$,  there exists $c,\rho> 0$ such that $c I-N$ and $N-\rho I$ are positive definite. Further, you can use Gronwall's Lemma, which states that for  differentiable functions $g,h:\mathbb{R}\rightarrow \mathbb{R}$ it holds that
\begin{equation*}
\frac{d g(t)}{dt}\leq h(t)g(t)~~ \forall t\in \mathbb{R} \implies g(t)\leq g(0) e^{\int_0^t h(s) ds},~~\forall t \in \mathbb{R}\,.
\end{equation*}
\end{itemize}
	
\begin{itemize}
\item[2.] $[$10 points$]$ As in the midterm: consider the sequence $\{x(k)\}_{k=0}^\infty \subseteq \mathbb{R}^n$ defined inductively by 
$x(k+1)=f(x(k))$ for some $f(\cdot):\mathbb{R}^n \rightarrow \mathbb{R}^n$, starting with some $x(0)\in \mathbb{R}^n$. The so-called \emph{Lyapunov's second method} for the discrete-time system $x(k+1)=f(x(k))$ states that if there exists some function $V(\cdot):\mathbb{R}^n \rightarrow \mathbb{R}^n$ such that $V(0)=0$, $V(x) > 0$ for all $x\neq 0$ and $\Delta V(x):=V(f(x)) - V(x)<0$ for all $x\neq 0$, then $\lim_{t \rightarrow \infty} x(t) =0$ for all $x(0) \in \mathbb{R}^n$. Based on this insight, derive the counterpart of the Lyapunov equation (2) for the sequence generated by $x(k+1)=Ax(k)$.
\end{itemize}

\begin{itemize}
\item[3.]  Salmons give birth when they reach a certain age $n \in \mathbb{N}$. The spawning process involves returning to their home-stream, making eggs and digging a nest. Since this requires a significant amount of energy, they die soon after, still at age $n$. Let $x_i(k)\in \mathbb{R}$ represent a measure quantifying the number of salmons of age $i$  alive at year $k$, where $k=1,\ldots,n$.  Let us denote the percentage of salmons of age $j$ that survive to the next year as $s_j \in (0,1)$, where $j=0,\ldots n-1$. Moreover, let $F\in \mathbb{R}_{>0}$ denote the fertility rate of salmons of age $n$, that is $Fx_n(t)$ new salmons are born every year (but $(100 \times s_0) \%$ of them immediately die). Specifically, we can write the system as follows:
\begin{equation*}
\begin{cases}
x_1(k+1)=s_0Fx_n(k)\,,\\
x_2(k+1)=s_1x_1(k)\,,\\
\vdots\\
x_n(k+1)=s_{n-1}x_{n-1}(k)\,.
\end{cases}
\end{equation*}
\begin{itemize}
\item[a.] $[$10 points$]$ Let $x(k)=\begin{bmatrix}x_1^\mathsf{T}(k),\ldots,x_n^\mathsf{T}(k)\end{bmatrix}^\mathsf{T}$. Using your result from part 2, determine conditions on the parameters $s_0,s_1,\ldots,s_n,F$ such that the salmons  go extinct, that is, $x(k)\rightarrow 0$ for $k\rightarrow \infty$. 

{\em Hint:} Letting $V(x(k))=x(k)^\mathsf{T}Px(k)$, find conditions such that there exists $P\succ 0$ that guarantees $$V(x(k+1)) - V(x(k))=-x(k)^\mathsf{T}x(k)\,, \quad \forall k \in \mathbb{N}\,.$$

\item[b.] $[$5 points$]$ Using your preferred method, determine a set of values $s_0,\ldots,s_{n-1},F$ such that salmons will not go extinct, but will not grow unbounded either. Explain your reasoning. Recall that  $0<s_j<1$ for every $j=0,\ldots,n-1$ and $F>0$.

\item[c.] $[$5 \textbf{bonus} points$]$ Assuming that $n=3$ and starting from an initial amount of salmons $x(0) \in \mathbb{R}^{3}$, plot the salmon trajectory over the years for choices of parameters such that i) salmons go extinct, ii) salmons do not go extinct and do not grow unbounded either, and iii) salmons grow unbounded. Explain your reasoning. 

\end{itemize}
\end{itemize}

\noindent {\bf Solution 1.1} \\

\noindent {\bf Solution 1.2} \\

\noindent {\bf Solution 1.3.a} \\

\noindent {\bf Solution 1.3.b} \\

\noindent {\bf Solution 1.3.c} \\

\clearpage

% exercise 2
\noindent {\bf Exercise 2, (LTV stability $[$30 points in total$]$) }

\begin{itemize}
\item[1.] $[$15 points$]$  Consider a time-varying matrix $A(t)$ in the form
\begin{equation}
A(t) = \begin{bmatrix}\lambda & \beta(t)\\ 0 & \lambda\end{bmatrix}\,, \tag{3}
\end{equation}
where $\lambda \in \mathbb{R}$ and $\beta\colon \R \to \mathbb{R}$ is given by $\beta(t) = ({1 - \lambda})\, e^{(1 - \lambda)\, t}$. Show that even if the eigenvalues of $A(t)$ in (3) have strictly negative real part,  the dynamical system $\dot{x}(t) = A(t)\, x(t)$ can be unstable.


\item[2.] $[$15 points$]$ Consider the linear time-varying system $\dot{x}(t) = A(t)\, x(t)$, where $A(t) = A^T(t) \in \R^{n \times n}$ and $\lambda(t) \leq -\epsilon < 0$ for all $t \geq t_0 \in \mathbb{R}$ and all $\lambda(t) \in \text{Spec}[A(t)]$. Show that this system is asymptotically stable.

{\em Hint:} Consider the function $V(x)=x^Tx$ and use Gronwall's Lemma (see Hint in Exercise~1 part~1).
\end{itemize} 

\noindent {\bf Solution 2.1} \\

\noindent {\bf Solution 2.2} \\

\clearpage

% exercise 3
{\bf Exercise 3, (Inner Product Spaces, $[$25 points in total$]$)}

Let $\mathcal{H}$ be a Hilbert space with inner product $ \langle \cdot, \cdot \rangle$, and let $\{ v_1, v_2, \ldots, v_n \}$ be an orthonormal collection of vectors in $\mathcal{H}$, i.e., $\left\| v_i \right\| = 1$ for all $i \in \{1, 2, ..., n\}$, and $\langle v_i, v_j\rangle = 0$ for all $j \neq i$.

\begin{itemize}
\item[1.] $[$15 points$]$ Prove that, for all $f \in \mathcal{H}$, 
$$ \sum_{ i=1 }^{n} | \langle  f, v_i \rangle |^2  \leq \left\| f \right\|^2.$$

\textit{Hint.} For all $f \in \mathcal{H}$, consider $g = \sum_{ i=1 }^{n} \langle f, v_i \rangle v_i \in \mathcal{H}$.

\item[2.] $[$10 points$]$ Assume now that $\mathcal{H}$ is finite dimensional and let $\{b_1, b_2, ..., b_n\}$ be an orthonormal basis for $\mathcal{H}$. Prove that, for all $f \in \mathcal{H}$,
$$ \sum_{ i=1 }^{n} | \langle  f, b_i \rangle |^2  = \left\| f \right\|^2.$$
\end{itemize}

\noindent {\bf Solution 3.1} \\

\noindent {\bf Solution 3.2} \\

\end{document}